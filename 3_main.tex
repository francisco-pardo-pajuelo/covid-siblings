
%%%%%%%%%%%%%%
% Introduction
%%%%%%%%%%%%%%

%Motivation x1
%Research question x1
%Method/Unique Features x1-2
%Findings x1-2
%Mechanisms/Robustness checks x2
%Contributions to Literature x3
%Policy Relevance/Big Picture x1



% 1. Motivation (1 Pargraph)

%\subsubsection{Motivation: 2 paragraphs}

The relationship between family size and children's educational outcomes has been a focus of interest in economics for many decades, with theoretical models suggesting a quantity-quality tradeoff in resource allocation (\cite{becker_child_1976}). While empirical evidence has challenged this simple tradeoff in normal circumstances (\cite{black_more_2005}), unexpected shocks that dramatically increase parental time and attention requirements may fundamentally alter this relationship (\cite{black_recent_2010}). When families face sudden increases in childcare responsibilities—particularly during critical developmental periods—the quality of education children receive can be severely affected if parents cannot adequately adapt their resource allocation strategies.

%The relationship between family size and quality of education given to each children has been long studied in economics. Models from \cite{becker_child_1976} and ... suggested a quantity-quality tradeoff although many empirical evidence has suggested otherwise (\cite{black_more_2005}). However, when faced with unexpected shocks to family size, parental time and financial resources, the quality of education that children receive might be affected, especially if they are exposed to it during longer periods or crucial developmental stages (\textcolor{green}{cite?}). These suggest that parents may be able to adapt to planned increases in family size (and childcare responsibilities) but if they are incapable of that, the quality of education that children receive is affected. Given that one of the main mechanisms through which this operate is parental time a financial constraints, we can expect it to have a significant relevance in a context that increasingly puts this to the test.

School closures during the COVID-19 pandemic provide a unique opportunity to examine family dynamics under extreme stress. In some countries these lasted for over a year, forcing household's role in education production to increase dramatically. Learning losses persist even after school reopenings, with severe impacts on vulnerable populations including low-income students and those without internet access (\cite{haelermans_inequality_2022}, \cite{jakubowski_global_2023}). In this context, parents who have multiple children may experience different challenges in adapting to the new circumstances. However, the role of family structure in amplifying these effects remains largely unexplored.\textbf{}

%Version 1
%The relationship between family size and children's educational outcomes has long captivated economists, with theoretical models suggesting a quantity-quality tradeoff in resource allocation (Becker and Tomes 1976). While empirical evidence has challenged this simple tradeoff in normal circumstances (Black, Devereux and Salvanes 2005), unexpected shocks that dramatically increase parental time and attention requirements may fundamentally alter this relationship. When families face sudden increases in childcare responsibilities—particularly during critical developmental periods—the quality of education children receive can be severely affected if parents cannot adequately adapt their resource allocation strategies.
%The COVID-19 pandemic created an unprecedented natural experiment in family dynamics when school closures worldwide transformed households into primary educational institutions. This transformation placed extraordinary demands on parental time and attention, with the household suddenly becoming the central actor in children's education production function. Documented learning losses persist even after school reopenings, with particularly severe impacts on vulnerable populations including low-income students and those without internet access (Haelermans et al. 2022; Jakubowski, Gajderowicz and Patrinos 2023). However, the role of family structure—specifically the presence of siblings—in amplifying these effects remains largely unexplored despite theoretical reasons to expect it would play a crucial role in determining how families managed this unprecedented shock.

%Version 2
%Educational production within families operates under normal circumstances through carefully balanced resource allocation, but what happens when external shocks fundamentally disrupt this equilibrium? The theoretical foundation from Becker and Tomes (1976) suggests that while families may successfully adapt to planned changes in size and structure, unexpected increases in childcare demands can overwhelm adaptive capacity, particularly when sustained over extended periods. The mechanisms through which this operates—primarily parental time and financial constraints—become especially relevant when considering contexts that stress-test these family systems.
%The global school closures during the COVID-19 pandemic provide a unique opportunity to examine family dynamics under extreme stress. When schools closed—some for weeks, others for over a year—the household's role in education production expanded dramatically. The resulting learning losses have been well-documented, with persistent negative effects particularly concentrated among disadvantaged populations (Haelermans et al. 2022; Jakubowski, Gajderowicz and Patrinos 2023). Yet despite extensive research on COVID-19's educational impacts, the differential effects based on family composition remain understudied. This gap is particularly striking given that sibling interactions and resource competition represent fundamental channels through which educational disruptions might be amplified or mitigated within households.

%%%%%%%%%%%%%%%%%%%%%%%%%%%%%
%2. Research question x1
%%%%%%%%%%%%%%%%%%%%%%%%%%%%%

%\subsubsection{Research Question: 1 paragraphs}
This paper explores how family structure mediates the impact of educational shocks during nationwide lockdown and school closures. How do sibling presence and dynamics affect educational outcomes when households suddenly assume a bigger role in education? Specifically, I examine whether children with siblings experienced differential learning losses compared to only children from all grades. This setting provides crucial insights into how families reallocate scarce resources when parental time and attention suddenly become more demanded and constrained, and how larger families may be systematically less able to adapt under such circumstances.

%%%###and beyond high school graduation through higher education
%%%###While the analysis focuses on an extreme situation, I believe

%MINE
%This paper makes use of administrative-level data from Peru to estimate the effect of a shock in childcare and education through nationwide lockdowns and school closures in the outcomes of children in the family. I use family size to study the exposure to this shock; specifically, I examine how different are the attainment and learning losses in children that have siblings compared to children who do not, measured from 1st to 11th grade, when they graduate from high school. Even though our estimation is based on a extreme situation, we believe it allows us to learn about how families reallocate resources when time and attention from parents is suddenly more required, and how bigger families may be less able to adapt under such circumstances. Sibling spillovers might occur through direct effects (mentoring or competing for resources like computers) or indirect effects through parental resource allocation, as increased childcare reduces time for educational activities. With reduced participation from schools and teachers in students’ education and children being at home, parental investments in education become more important but more constrained due to increased childcare responsibilities; hence, these sibling spillovers may be exacerbated.

%Version 1
%This paper leverages administrative data from Peru to estimate how family structure mediates the impact of educational shocks during nationwide lockdowns and school closures. Specifically, we examine whether children with siblings experienced differential learning losses compared to only children, measured across grades 1-11 through high school graduation. While our analysis focuses on an extreme situation, we believe this setting provides crucial insights into how families reallocate scarce resources when parental time and attention suddenly become more constrained, and how larger families may be systematically less able to adapt under such circumstances.
%Our research design exploits the variation in exposure to childcare shocks created by family size differences. When schools closed and children remained home, parents faced dramatically increased responsibilities for both education and supervision. Sibling spillovers likely operated through both direct channels—such as competition for educational resources like computers or parental attention—and indirect channels through the reallocation of parental time away from educational activities toward basic childcare. With reduced institutional support from schools and teachers, parental educational investments became simultaneously more important and more constrained, potentially exacerbating these sibling spillover effects.

%Version 2
%We examine a fundamental question about family resource allocation under stress: how do sibling dynamics affect educational outcomes when households suddenly become primary educational institutions? Using comprehensive administrative data from Peru spanning pre-kindergarten through 11th grade, we identify the causal impact of family size on learning losses during the COVID-19 school closures by comparing educational outcomes of children with siblings to only children before and after this unprecedented shock.
%Our identification strategy recognizes that while the number of children in a family is not randomly distributed, we can leverage the differential exposure to increased childcare demands that school closures created. Even though both only children and children with siblings experienced significant increases in required parental involvement, families with multiple children faced proportionally larger shocks to their time and attention resources. This allows us to estimate how much more severely families with multiple children were affected, providing insights into the marginal effects of increased childcare responsibilities on educational outcomes and revealing how sibling spillovers operate when family resources are stretched to their limits.


%3. Method/Unique Features x1-2

%\subsubsection{Method/Unique Features: 3 paragraphs}

I use administrative data on school progression and performance from Peru's education system, identifying siblings using anonymized parent IDs across all enrolled students from pre-K through 11th grade. This dataset is complemented by standardized national examinations administered in 2nd, 4th, and 8th grades, allowing me to demonstrate effects beyond grade point averages using comparable achievement measures. The standardized tests were frequently coupled with detailed parent and student surveys that enable rich heterogeneity analysis across household resources, parental time investments, socioeconomic status, and educational expectations. Additionally, I extend the analysis beyond immediate schooling outcomes by matching these data with college applications and enrollment records to study longer-term educational trajectories.

Our identification strategy addresses endogeneity concerns by comparing families with multiple children, who arguably experienced larger childcare shocks, to those with only one child using a difference-in-differences framework. Although these groups have some differences in their performance, they exhibited parallel trends before school closures, allowing me to estimate how differentially they were affected by varying exposure to \textcolor{black}{parental time requirement shocks}. I specifically compare outcomes of first-born children in multi-child families (\textit{more exposed}) versus only children (\textit{less exposed}), eliminating biases from time-invariant family characteristics. 

I strengthen this identification framework through two complementary quasi-experimental strategies. First, I exploit discontinuous variation generated by school starting age cutoffs, which determine whether younger siblings would normally attend school or remain home. This policy-induced variation creates an exogenous shock to household childcare demands: families with younger siblings born just before enrollment cutoffs experience a sudden increase in home supervision requirements during closures, while those with siblings born just after cutoffs face relatively smaller disruptions. Second, I implement a difference-in-differences design comparing academic performance in the year of a sibling's birth versus the previous year, when parents arguably had greater time availability for educational support. I hypothesize that this performance gap widens during school closures, when reductions in parental investment in older siblings due to caring for a newborn have bigger consequences.


%%%%% MINE
%I use administrative data on school progression and performance from Peru and identify siblings by using anonymized parent's IDs. These include all enrolled students from pre-K to 11th grade. Additionally, we compliment this with standardized national exams taken in 2nd, 4th and 8th grade that allows me to show effects beyond GPA. These examinations were often coupled with parent and student surveys that compliment the heterogeneity analysis through household resources, parental time and invetment, socio-economic status, education expectations and others. Finally, I match this data with college applications and enrollment to study effects beyond school outcomes.

%To address the identification issue, we compare families with multiple children, that arguably experience a larger shock in childcare requirements, to those with only one child. Because the number of children is not randomly distributed across families, we look at the differences across time.  Even though these two groups are different, they have parallel trends before school closures and in that regard, our estimate is how differentially affected they have been as a consequence of the difference in exposure to shocks in parental time requirements. I am thus comparing outcomes of the first-born in families with multiple children (more affected) and those where they are the only children (less affected) before and after school closures. In doing so, I eliminate biases associated with unobserved time-invariant family characteristics.

%While this approach allows us to estimate the effects of relative increases in childcare, we are unable to estimate the total effect of increases in childcare due to school closures. Given that families with only one children also experience significant increases in the parental time required to care and educate their child at home, our estimate is just how much more are families with multiple children affected by this. However, we can see the increases with the number of siblings as potential marginal effects of increased childcare responsibilites. %Another limitation of our research design is that it relies on the assumption that trends....

%Additionally, to address concerns about potential issues when comparing students with different amount of siblings. I also leverage two other potential sources of variation. First, I explore school starting age cutoffs and sibling spillovers. When schools operate normally, having a younger sibling born before the cutoff meant having them go to school early and release parents from some of the care required of a 5-6 year old. During school closures, that same younger sibling would have to stay in the house, potentially diluting parental resources and affecting their other siblings. Second, I do a difference in difference across time and birth of a sibling, comparing performance across time on the year of birth of a sibling and the previous one, where parent arguably had more time. \textcolor{green}{Hypothesis: I find that this gap is bigger during school closures, evidencing that effect of having a 0-1 year old child to take care of.}

%Our analysis utilizes comprehensive administrative data on school progression and performance from Peru's education system, identifying sibling relationships through anonymized parent IDs across all enrolled students from pre-K through 11th grade. This dataset is complemented by standardized national examinations administered in 2nd, 4th, and 8th grades, allowing us to demonstrate effects beyond grade point averages using comparable achievement measures. The standardized tests were frequently coupled with detailed parent and student surveys that enable rich heterogeneity analysis across household resources, parental time investments, socioeconomic status, and educational expectations. Additionally, we extend our analysis beyond immediate schooling outcomes by matching these data with college applications and enrollment records to study longer-term educational trajectories.
%Our identification strategy addresses endogeneity concerns by comparing families with multiple children—who arguably experienced larger childcare shocks—to those with only one child using a difference-in-differences framework. Although these groups differ systematically, they exhibited parallel trends before school closures, allowing us to estimate how differentially they were affected by varying exposure to parental time requirement shocks. We specifically compare outcomes of first-born children in multi-child families (more affected) versus only children (less affected), eliminating biases from time-invariant family characteristics. To strengthen identification, we exploit two additional sources of variation: school starting age cutoffs that determined whether younger siblings were in school or at home during closures, and temporal variation around sibling births to identify moments when parental attention was particularly constrained.



%%%% FROM SANDY
% While this approach allows us to estimate the effects of relative exposure to a disabled sibling, we are unable to estimate the total effect of exposure to a disabled sibling. \textbf{Because families with a disabled child may be different in unobserved ways from other families, we are unable to distinguish whether differences in outcomes are the result of exposure to a disabled child or of differences in unobservables across these families.}

%A key underlying assumption of our identification strategy is that the effect of birth order does not vary based on the presence of a disabled third sibling, other than through the differential exposure to the disabled third sibling. While we cannot test this directly, we do conduct a number of exercises to suggest that this might be a reasonable assumption.
%Because our research strategy is based on differential exposure to disabled younger siblings in early life, it is particularly important to observe disabilities that are noticeable to families early. We make use of administrative data that reveal when children first obtain services for their disabilities, and we concentrate on disabilities that are first identified in or before kindergarten. It is harder to separate the spillover effect from the common-shock effect for disabilities first diagnosed in the elementary school years because disabilities diagnosed later are more likely to reflect unobserved family factors rather than a singular health or physical shock. Furthermore, as an alternative to measuring the effects of disabilities that appear in early childhood, we also consider the spillover effects of exposure to a third-born sibling with health problems at birth, such as congenital anomalies, abnormal conditions, low birth weight, prematurity or very poor perinatal health. We find similar results regardless of whether we study the spillover effects of the third-born child’s disability as recorded in early childhood or of the child’s poor birth conditions.
%We estimate these relationships using data from two data sources, drawn from two distinct locations: Florida and Denmark. In both cases, we make use of matched administrative data sets where we merge full birth cohorts with schooling and medical data. The large sample sizes allow us to generate good statistical power while focusing on families with at least three children, a subset of which has a third child with disabilities. The data also allow us to follow children for whom we know birth outcomes and family structure into early adolescence to measure cognitive development.
%The Florida data include birth records for every birth in the state between 1994 and 2002, merged with school records for all children who attended public schools within the state. The key outcome we measure in Florida is the state’s criterion-referenced end-of-year standardised test, the Florida Comprehensive Assessment Test (FCAT), administered to children in grades 3–8 over the relevant time period. (For more detail on these data, see Figlio et al., 2013 and Figlio et al., 2014.)
%The Danish data include birth records for every birth in the country between 1990 and 2001, merged with medical patient register and school records. Our main outcome of interest in Denmark is the cohort-standardised grade from the 9th grade exit exam (GPA). These exam results are based on assessments by the student’s teacher and by an external reviewer. While these exam scores do not capture exactly the same features as the Florida exams, they are widely interpreted to represent similar types of outcomes in the extant literature.
%Including data from two very different contexts provides a number of advantages. For one, the United States and Denmark have very different educational systems, healthcare systems and social welfare systems. To the degree to which we find comparable results in these two quite different environments, it further increases our confidence that the estimates are externally valid and more generally reflect spillovers and dynamics within the family, as opposed to something that is very context-specific. Second, disability is measured differently in the two contexts—on the education record in Florida and on the medical record in Denmark. This fact also aids in establishing the degree to which our findings are externally valid. Furthermore, having data from both settings provides greater opportunities to explore the mechanisms through which these findings might be operating because the two countries have quite different population characteristics and different data strengths. Thus, between the two sites, we are able to paint a more complete picture about the nature of the sibling spillovers that we seek to investigate.


%4. Findings x1-2

%\subsubsection{Findings: 2 paragraphs}

I find that students with siblings experienced significantly larger learning losses than only children across the full spectrum of academic measures. These differential effects are remarkably consistent across diverse subpopulations—appearing in both rural and urban areas, among low and high socioeconomic status families, and across varying levels of parental education. The magnitude of the sibling effect increases with the number of siblings, and the impacts are consistently more pronounced in elementary grades where parental input plays a more crucial role in academic development.

My results extend beyond immediate test scores to encompass broader educational trajectories. Using survey data on educational expectations, I find that parents of children with siblings became systematically more pessimistic about their children's long-term educational prospects, with decreased expectations for four-year college completion and increased predictions that high school would represent the maximum educational attainment. These expectation changes suggest that the immediate learning losses may have persistent effects on family educational investments even after schools reopen. 

%%%%###
%\textcolor{green}{The analysis of college application and enrollment data reveals persistent effects on post-secondary educational choices.} \textcolor{black}{Children with siblings showed greater declines in grade progression rates and increased likelihood of academic setbacks during the closure period.}


%Version 1
%We document that students with siblings experienced significantly larger learning losses than only children across the full spectrum of academic measures. These differential effects are remarkably consistent across diverse subpopulations—appearing in both rural and urban areas, among low and high socioeconomic status families, and across varying levels of parental education. The magnitude of the sibling effect increases with the number of siblings, and the impacts are consistently more pronounced in elementary grades where parental input plays a more crucial role in academic development.
%Our results extend beyond immediate test scores to encompass broader educational trajectories. Children with siblings showed greater declines in grade progression rates and increased likelihood of academic setbacks during the closure period. Using survey data on educational expectations, we find that parents of children with siblings became systematically more pessimistic about their children's long-term educational prospects, with decreased expectations for four-year college completion and increased predictions that high school would represent the maximum educational attainment. These expectation changes suggest that the immediate learning losses may have persistent effects on family educational investments even after schools reopen.

%Version 2
%Students from multi-child families suffered disproportionately larger learning losses compared to only children, with effects that scale systematically with family size. Children with one sibling experienced moderate additional losses, while those with two or three siblings faced increasingly severe educational setbacks. These patterns emerge consistently across both standardized test scores and administrative grade records, indicating that the effects transcend specific measurement approaches. Importantly, the differential impacts are most pronounced in elementary schools, where parental time and attention traditionally play more central roles in educational production, supporting our hypothesized mechanisms.
%The consequences extend well beyond immediate academic performance to encompass fundamental changes in educational trajectories and family expectations. Grade progression rates declined more sharply for children with siblings, and our analysis of college application and enrollment data reveals persistent effects on post-secondary educational choices. Survey evidence indicates that parents of children with siblings substantially revised downward their educational expectations during the crisis, with significant increases in the likelihood of viewing high school as the terminal educational goal rather than expecting college completion. These expectation adjustments suggest that the crisis may have lasting implications for human capital investment decisions within affected families, potentially perpetuating the immediate educational losses into long-term outcomes.


%5. Mechanisms/Robustness checks x2

%\subsubsection{Mechanisms/Robustness: 2 paragraphs}

I explore alternative explanations to isolate the mechanisms driving my main results. First, to distinguish family size effects from birth order effects, I focus on first-born children when comparing only children with children who have siblings. Second, I investigate whether siblings directly disrupted each other through shared study environments or competition for technological resources. However, when focusing on families with large age gaps between siblings and households without computers or internet access results are similar, suggesting that direct sibling interference is not the primary mechanism. Third, the presence of siblings can also have indirect effects through their effect on parental time dilution. The evidence points strongly toward these indirect effects operating through parental time constraints as the dominant channel. My findings are consistently larger among elementary school children, where parental time investments are most critical for development. Using the school starting age discontinuity, I demonstrate that having a younger child remain home rather than attend school significantly reduces older siblings' academic performance and measured parental investment. Results vary systematically with parental education levels and family structure in ways consistent with time constraint mechanisms. Finally, larger families could also experience larger income shocks. However, I find no evidence of this during school closures, and my effects are actually strongest among high socioeconomic status families who are best equipped to handle financial losses but typically invest more time in their children's education.

%%%MINE
%I explore different potential mechanisms behind these effects. First, to avoid confounding birth order effects with family size effects, I focus on the first-born children both when comparing only children with children with siblings. Second, siblings could be directly affecting each other. We could expect closely spaced siblings to be more disruptive to each other when sharing a study environment, however, results are similar when the age gap between siblings is larger. Also, one could expect siblings fighting for a computer during remote schooling, however, results are similar in households without PC and internet than those who had them. Third, siblings might have indirect effects, mainly through their effect on parental time: not being able to leave them in childcare, pre-K, or school, parents now need to not only take up on increased childcare responsibilities, an increase that is bigger the more children they have, but also on schooling ones. Most evidence points in this direction: (i) results are larger in children from elementary school, where parental time investments are more important for development, (ii) leveraging school starting ages (SSA) I find that having a child stay in the household rather than going to school decreases older siblings' scores and parental investment (iii) Results when focusing on parents by level of education and living situation .... (\textcolor{green}{Check also with ECE}. Fourth, bigger families might experience a larger income shock from lockdowns which may as a consequence affect children from those families more. However, we do not find significant effects in changes in socioeconomic status of those families. I also find robust effects in both low and high income families, and evene larger ones in families from the top quartile of SES, which are more equipped to deal with income losses and at the same time spend more time with their children (\textcolor{green}{reference})

%Version 1
%We systematically explore alternative explanations to isolate the mechanisms driving our main results. First, to distinguish family size effects from birth order effects, we focus our analysis on first-born children when comparing only children with children who have siblings, eliminating confounding from ordinal position within families. Second, we investigate whether siblings directly disrupted each other through shared study environments or competition for technological resources. However, we find similar results when examining families with large age gaps between siblings and when analyzing households without computers or internet access, suggesting that direct sibling interference is not the primary mechanism.
%The evidence points strongly toward indirect effects operating through parental time constraints as the dominant channel. Our findings are consistently larger among elementary school children, where parental time investments are most critical for development. Using the school starting age discontinuity, we demonstrate that having a younger child remain home rather than attend school significantly reduces older siblings' academic performance and measured parental investment. Results vary systematically with parental education levels and family structure in ways consistent with time constraint mechanisms. Importantly, we find no evidence that larger families experienced systematically different income shocks during lockdowns, and our effects are actually strongest among high socioeconomic status families who are best equipped to handle financial losses but typically invest more time in their children's education.

%Version 2
%Our mechanism analysis reveals that parental time constraints, rather than direct sibling interactions or income effects, primarily drive the observed learning loss differentials. We first address birth order confounds by restricting our analysis to first-born children, ensuring that family size effects are not contaminated by sibling position dynamics. Testing for direct sibling disruption through shared learning environments yields little support—results remain robust when examining families with widely spaced siblings and households lacking educational technology, suggesting that resource competition for computers or study space is not the primary channel.
%Multiple pieces of evidence converge on parental time allocation as the central mechanism. The school starting age regression discontinuity provides quasi-experimental evidence that having additional children at home significantly reduces both academic outcomes and parental educational investments for older siblings. The effects are systematically larger in elementary grades where parental involvement is most crucial, and vary predictably with family characteristics that proxy for time availability and educational engagement. Strikingly, we observe the strongest effects among high socioeconomic status families—those most capable of weathering income shocks but also most likely to provide intensive educational support to their children. This pattern, combined with the absence of differential income effects across family types, strongly supports the interpretation that constraints on parental time and attention, rather than financial resources, drove the differential impacts we observe.

%6. Contributions to Literature x3

%\subsubsection{Contribution to literature: 3 paragraphs}

My results contribute to three strands of research. First, it relates to the literature of family structure and quality of education. Whether there is a quantity-quality tradeoff in the amount of children and the level of education they receive has been largely studied in economics (\cite{becker_child_1976}, \cite{black_more_2005}, \cite{black_small_2010}). Research has shown that this is not always the case but that rather unexpected shocks can cause this tradeoff to exist. For instance, parents can plan and adapt to having a new child in a way that does not affect the quality of education received by other children but having twins may alter that balance. In a similar way, school closures act as an unexpected shock in increased childcare required from parents as well as increased time invested in education. This may cause a tradeoff to arise in larger families.

Second, it relates to the literature of health shocks and spillover effects in the family. Some of this work explains how increased parental time required by a child who experiences an adverse event can cause negative effects on the rest of the children. \cite{black_sibling_2021} finds that having a disabled child has negative spillovers on educational outcomes of their siblings.

Third, it relates to the literature about learning losses from the school closures led by the COVID-19 pandemic. There is ample evidence of large negative and persistent effects caused by school closures that are larger for more vulnerable populations(\cite{haelermans_inequality_2022}, \cite{jakubowski_global_2023}). My results show a different aspect that sheds light on how part of this learning losses occurred: through the increased difficulties of translating education into the households when there are multiple children and limited parental time. Research from other fields have found positive effects of having siblings during the pandemic. \cite{hughes_siblings_2023}, \cite{lampis_long-lasting_2023} show that siblings act as a buffering given the loss of other peers which lead to better linguistic and emotional-behavioral outcomes. Some of this though is early in the pandemic and not focused on educational outcomes. The potential multidimensional aspect of these effects is however an interesting area that requires further research.


%that applying their findings the negative effect of unexpected changes into another context that requires much more parental involvement that would otherwise be covered by schools ... \cite{black_small_2010}.... Second, it relates to the literature of health shocks ...\cite{black_sibling_2021}... Third, it relates to the literature of COVID effects on education.... Fourth, it relates to literature on the interaction between parental and school inputs. \cite{bonesronning_determinants_2004} finds that parents respond to more teacher efforts per student by increasing their own efforts although only for smaller class sizes (or high levels of teacher effort). Our findings go against this, potentially highlighting a U-shaped relation between parental and teacher effort.


%\cite{adhvaryu_endowments_2016} find that siblings of treated children were also more likely to be immunised. $\rightarrow$ In favor of parents equalising. Not sure the model is what we need though. \cite{yi_early_2015} also in favor of equalisers. Can I place myself in this literature? How do I estimate one of these? Perhaps if I see that low performers are not the ones doing worse?

\begin{comment}
Other fields: \cite{hughes_siblings_2023}, \cite{lampis_long-lasting_2023} show that siblings act as a buffering during the pandemic. With better linguistic and emotional-behavioral outcomes. Mid 2020 and late 2021. 

Review: 
\cite{behrman_chapter_1997}
\cite{behrman_parental_2022} 
\cite{attanasio_early_2022}

Health Shocks:
\cite{bharadwaj_health_2018}
\cite{yi_early_2015}

Time Use: 
\cite{conti_parental_2022} (also beliefs)

Sibling Inequality
\cite{giannola_parental_2024}

Models
\cite{cunha_technology_2007}
\cite{rosenzweig_heterogeneity_1988}
\cite{behrman_parental_1982}: 
\end{comment}




%Version 1
%Our findings contribute to several important research streams. First, we advance the literature on family size effects by demonstrating how unexpected shocks can activate the quantity-quality tradeoffs predicted by theory (Becker and Tomes 1976) even when they may not operate under normal circumstances. While previous research has found mixed evidence for family size effects in typical settings (Black, Devereux and Salvanes 2005), our study shows that resource constraints become binding when families face extraordinary demands on parental time and attention. This provides crucial evidence for understanding when and why family structure matters for educational outcomes.
%Second, our work contributes to the emerging literature on health and educational shocks within families (Black et al. 2021). By demonstrating how educational disruptions propagate differently across family structures, we show that the impacts of policy interventions or external shocks may have important spillover effects that depend on household composition. This has significant implications for predicting the distributional consequences of educational policies and for designing interventions that account for family dynamics.
%Third, our study adds to the growing body of research on COVID-19's educational impacts by identifying family structure as a crucial but previously overlooked moderating factor. While existing studies have documented learning losses and their relationship to socioeconomic status and technology access (Haelermans et al. 2022; Jakubowski, Gajderowicz and Patrinos 2023), our research reveals that family composition represents an independent and substantial source of inequality in pandemic impacts, with implications for long-term educational trajectories.

%Version 2
%This research makes several novel contributions to economic understanding of family dynamics and educational production. We provide new evidence on the quantity-quality tradeoff in human capital investment by showing how resource constraints become binding under stress, even when they may not operate under normal circumstances. Our findings reconcile theoretical predictions from Becker and Tomes (1976) with mixed empirical evidence from previous studies (Black, Devereux and Salvanes 2005) by identifying the conditions under which family size effects emerge: when external shocks dramatically increase the demands on family resources beyond their adaptive capacity.
%We extend the literature on sibling spillovers and family resource allocation by documenting how educational shocks propagate through family networks. Our evidence complements recent work on spillovers in college and career choices (Altmejd et al. 2021) by demonstrating that sibling effects operate not only through information transmission and role modeling, but also through resource competition during periods of heightened family stress. This represents an important addition to our understanding of how family dynamics shape human capital investment decisions across different contexts and life stages.
%Finally, our study contributes to the COVID-19 education literature by identifying family structure as a fundamental but previously unrecognized dimension of inequality in pandemic impacts. While existing research has focused on socioeconomic status, technology access, and geographic factors as determinants of learning losses, we show that family composition represents an independent and substantial source of differential impacts. This finding has important implications for understanding the long-term consequences of the pandemic and for designing policies to address its educational effects, particularly given that family size correlates with many other socioeconomic characteristics in ways that could compound disadvantage.


%\subsubsection{Policy relevance/Big picture: 1 paragraphs}

%7. Policy Relevance/Big Picture x1
These results shed light on the potential mechanisms that can drive learning losses when families face unexpected shocks that both constraint and increase the demand for time investments in the education of their children by showing how larger families can be more vulnerable in such circumstances. Furthermore, this is not something that families with higher resources may be able to overcome given the nature of the resource constraint that is driving the results: parental time. Extending our results to other settings, this can be causing persistent effects in several countries, developing and developed, particularly those who had longer periods of school closures also across countries (\hyperref[fig:pisa]{Figure \ref{fig:pisa}}), especially those most affected by school closures.

%Version 1
%Our findings have significant implications for educational policy and crisis response strategies. The persistence of effects we document suggests important opportunities for targeted interventions among affected populations, particularly families with multiple children who experienced the most severe learning losses. Since our results demonstrate that these effects are widespread across countries with extensive school closures, the policy relevance extends well beyond Peru to inform global discussions about pandemic recovery and future crisis preparedness in education systems.
%The identification of family structure as an independent source of educational inequality during the pandemic highlights the need for policies that account for household composition when designing educational interventions. Traditional approaches that focus solely on individual student characteristics or school-level factors may miss crucial family-level dynamics that determine the effectiveness of educational investments. Our findings suggest that crisis response strategies should explicitly consider family size and structure when allocating resources and designing support programs, potentially providing additional assistance to larger families during periods of educational disruption to prevent the amplification of existing inequalities through family spillover effects.

%Version 2
%The policy implications of our research extend far beyond the immediate COVID-19 context to fundamental questions about educational equity and family support systems. Our demonstration that family structure independently affects educational outcomes during crises suggests that current policy frameworks may systematically under-serve larger families during periods of educational disruption. This insight is particularly relevant for designing future crisis response strategies and for understanding how educational policies interact with family dynamics to produce unintended distributional consequences.
%More broadly, our findings reveal the existence of social multiplier effects within families that can either amplify or mitigate the impacts of educational interventions. Just as older siblings can influence younger siblings' college choices (Altmejd et al. 2021), our evidence suggests that educational shocks to any child can have cascading effects throughout the family. This implies that the true costs and benefits of educational policies may be larger than typically estimated in studies focused on directly treated students, as interventions can have ripple effects through family networks. Understanding these dynamics is crucial for designing more effective and equitable educational policies that account for the complex ways families respond to external shocks and policy changes.



% 6. Roadmap of paper.
The paper proceeds as follows. \hyperref[sec:empirical_strategy]{Section \ref{sec:empirical_strategy}} details the empirical approach. \hyperref[sec:data]{Section \ref{sec:data}}  describes the higher education system in Peru and linked administrative data. \hyperref[sec:results]{Section \ref{sec:results}} presents results. \hyperref[sec:mechanisms]{Section \ref{sec:mechanisms}} discusses potential mechanisms. \hyperref[sec:conclusions]{Section \ref{sec:conclusions}} concludes.
%\hyperref[sec:heterogeneity]{Section \ref{sec:heterogeneity}} discusses heterogeneity in results. \hyperref[sec:robustness]{Section \ref{sec:robustness}} shows robustness checks. \hyperref[sec:policy]{Section \ref{sec:policy}} discusses policy implications. 

%Version 1
%The paper proceeds as follows. Section I details our empirical approach and identification strategy. Section II describes Peru's educational system and our linked administrative datasets. Section III presents our main results on learning losses and educational trajectories. Section IV discusses potential mechanisms underlying our findings. Section V examines heterogeneity across different subpopulations and contexts. Section VI presents robustness checks and alternative specifications. Section VII discusses policy implications and broader significance. Section VIII concludes.

%Version 2
%The remainder of this paper is organized as follows. Section I outlines our identification strategy and empirical methodology. Section II provides institutional background on Peru's education system and describes our comprehensive administrative data. Section III presents results on differential learning losses by family structure. Section IV investigates mechanisms through which sibling effects operate during school closures. Section V explores heterogeneous effects across socioeconomic groups and family characteristics. Section VI conducts robustness tests and addresses potential threats to identification. Section VII discusses policy implications and broader contributions. Section VIII concludes with directions for future research.


%###

\begin{comment}

\newpage

\hyperref[fig:pisa]{Figure \ref{fig:pisa}}

\hyperref[fig:trend]{Figure \ref{fig:trend}}

\hyperref[fig:main_result]{Figure \ref{fig:main_result}}

\hyperref[tab:descriptive]{Table \ref{tab:descriptive}}

\hyperref[tab:twfe_ece]{Table \ref{tab:twfe_ece}}

\hyperref[tab:twfe_gpa_baseline_survey_1_pairall]{Table \ref{tab:twfe_gpa_baseline_survey_1_pairall}}

\hyperref[tab:twfe_gpa_baseline_survey_2_pairall]{Table \ref{tab:twfe_gpa_baseline_survey_2_pairall}}


\hyperref[tab:rd_summ_1_m_a_365]{Table \ref{tab:rd_summ_1_m_a_365}}

\hyperref[tab:rd_ece_index_365]{Table \ref{tab:rd_ece_index_365}}


% Appendix

\hyperref[fig:trend_pass_grades]{Figure \ref{fig:trend_pass_grades}}

\hyperref[fig:trend_gpa_grades]{Figure \ref{fig:trend_gpa_grades}}




\hyperref[fig:fig_appA]{Figure \ref{fig:fig_appA}}

\hyperref[fig:fig_appB]{Figure \ref{fig:fig_appB}}

\hyperref[fig:fig_appC]{Figure \ref{fig:fig_appC}}

\hyperref[fig:fig_appD]{Figure \ref{fig:fig_appD}}



\hyperref[tab:rd_summ_K_m_a_365]{Table \ref{tab:rd_summ_K_m_a_365}}

\hyperref[tab:rd_summ_2_m_a_365]{Table \ref{tab:rd_summ_2_m_a_365}}



\hyperref[tab:twfe_gpa_baseline_survey_1_pair1]{Table \ref{tab:twfe_gpa_baseline_survey_1_pair1}}


\hyperref[tab:twfe_gpa_baseline_survey_1_pair2]{Table \ref{tab:twfe_gpa_baseline_survey_1_pair2}}

\hyperref[tab:twfe_gpa_baseline_survey_1_pair3]{Table \ref{tab:twfe_gpa_baseline_survey_1_pair3}}

\hyperref[tab:twfe_gpa_baseline_survey_1_pair4]{Table \ref{tab:twfe_gpa_baseline_survey_1_pair4}}

\hyperref[tab:twfe_gpa_baseline_survey_2_pair1]{Table \ref{tab:twfe_gpa_baseline_survey_2_pair1}}

\hyperref[tab:twfe_gpa_baseline_survey_2_pair2]{Table \ref{tab:twfe_gpa_baseline_survey_2_pair2}}



\hyperref[tab:twfe_gpa_baseline_survey_2_pair3]{Table \ref{tab:twfe_gpa_baseline_survey_2_pair3}}


\hyperref[tab:twfe_gpa_baseline_survey_2_pair4]{Table \ref{tab:twfe_gpa_baseline_survey_2_pair4}}

\end{comment}

%%%%%%%%%%%%%%
% Data
%%%%%%%%%%%%%%
\section{Data}\label{sec:data}
I estimate the effects of family structure on educational outcomes before, during and after school closures using comprehensive school administrative and survey data matched with administrative college applications and enrollment. These data spans from 2014 to 2024 allowing me to explore effects even after schools re-opened

\subsection{Sibling Identification}

Our definition of exposure to school closures is based on family structure, or more precisely, on the number of children in the family. In order to do that, I use anonymized parent IDs to identify students who have the same mother as a proxy of them living together. %I have this information for \textcolor{black}{98\%} of those enrolled from Pre-K to 11th grade. \footnote{There are potential concerns with this identification of siblings: (i) It relies on students being enrolled. According to household surveys, \textcolor{green}{XX\%} of students are enrolled in school. (ii) There is some potential bias given that some students born in 2020 might still not be observed by 2024 and the sample of only children in that year might be overrepresented. I discuss this in Appendix \textcolor{green}{XX} by using an alternative measure of sibling identification, one that relies on information available only until 2021. (iii)... Furthermore the average household size is \textcolor{green}{XX} which is close to household survey estimates of \textcolor{green}{XX\%}.} 
%%%### Relevant footnote

In \hyperref[tab:descriptive]{Table \ref{tab:descriptive}}, I show the means for the second grade population of first-born children during years 2015-2016, to provide a comparable population between the administrative data and the survey information from panels D, E and F. I show that 57\% of first-born students are only children, 27.1\% have one younger sibling, 12.2\% have 2 younger siblings and 3.7\% have 4 younger siblings. Overall, only children are very similar in their characteristics to children with one sibling. Children with 2 or 3 siblings are less similar, with them having higher rates of rural areas, public schools and overall lower socio-economic characteristics as see by their access to resources and parental education. Also, they have significantly lower academic performance and parental expectations.


\subsection{Administrative data on school progression and GPA (SIAGIE)}

From 2014 to 2024, I have access to data for all students enrolled in the schooling system from pre-k to 11th grade, in public and private schools. These data has information on academic grades by subject and overall grade progression. Additionally, it has information on the school characteristics, sex, parent's level of education and date of birth of students and parents. In \hyperref[tab:descriptive]{Table \ref{tab:descriptive}}, I show that 77.3\% of students from the only children sample come from urban areas and 93.1\% are promoted from second to third grade. There are some differences in the average GPA, with the children with one sibling having higher averages than the rest and those with three siblings having lower levels. Also, in \hyperref[fig:trend]{Figure \ref{fig:trend}} I show that both populations, although at different levels of academic achievement, had parallel trends before school closures and after that, an additional gap emerges.

\subsection{Standardized National Examinations (ECE)}

Students were tested through standardized tests in specific grades and years. \footnote{Second grade students were tested from 2007 to 2016 nationally and then in smaller samples in in 2019 and 2022. Fourth grades students were tested nationally in 2016, 2018 and 2024 and in smaller samples in 2019, 2022 and 2023. Eight grade students were tested nationally in 2015, 2016, 2018 and 2019 and in smaller samples in 2022 and 2023.} This allows a measure of learning losses that is not dependent on within school-grade variations. These tests are standardized with mean 0 and standard deviation of 1 in the base year of 2007, in order to have comparable measures across time. \footnote{Exams are scaled across years based on a control sample that takes both.} In \hyperref[tab:descriptive]{Table \ref{tab:descriptive}}, I show that only children score 0.067 standard deviations lower than children with one sibling in standardized mathematics examinations and 0.058 standard deviations higher than children with two siblings. The percentage of students who did 3 years of pre-k is similar between these groups.  

\subsection{Surveys}

In some of the years were students were tested by ECE, a survey was taken to teachers, principals, parents (in 2nd and 4th grade) and students (in 8th grade). These include information from socio-economic status, parent's mother tongue, expectations for educational attainment, parental investment in education, access to internet and a PC, number of bedrooms, etc. In \hyperref[tab:descriptive]{Table \ref{tab:descriptive}}, I show that most parents have high expectations for the maximum level of education that their children will achieve. 79.1\% of parents of only children expect that to be college education or higher, similar to 81.7\% of parents of children with one sibling.




%\textcolor{green}{Table of descriptive statistics by family size}
%\textcolor{green}{Figure of trends. Use actual GPA instead of standardized?}

%%%%%%%%%%%%%%
% Empirical Strategy
%%%%%%%%%%%%%%

\section{Empirical Strategy}\label{sec:empirical_strategy}

My research approach is to carry out a simple difference-in-differences design: I compare first-born children from families with one versus multiple children. The basic idea underlying the research design is that, because of school closures, children remained at home and a lot of the burden of education relied on parents monitoring and spending more time with their children which given the limited time they had, meant that families with more children had less time to invest in each of them when there was no in-person school to substitute for that reduced investment. However, first-born children from families with different sizes may have different outcomes. I therefore look at the variation between these two groups over time as a way of separating the family structure effect from the effect of differences in exposure to the school closures. Put differently, the comparisons in \hyperref[tab:descriptive]{Table \ref{tab:descriptive}} make it clear that families with one children are different in some ways from families with multiple children, especially as the number of children increases. My research design accounts for those differences across families by making comparisons between them across time.

Our main regression equation is the following:

    \begin{align}
    Y_{isgt} &= \alpha + \delta_1 \text{Post}_{it} + \delta_2 \text{S}_{i}   + \boldsymbol{\beta} \text{Post}_{it} \text{S}_{i}  + \gamma\mathbf{X}_{ist} + \lambda_s + \mu_g + \tau_t + \varepsilon_{isgt}
    \end{align}

     \begin{align}
    Y_{isgt} &= \alpha + \delta_1 \text{Post}_{it} + \delta_2 \text{S}_{i}   + \sum_{k=-5}^{-2} \delta_k (\mathbb{I}[t = 2020+k] \text{S}_{i}) \nonumber \\
    & \quad + \sum_{k=0}^{4} \beta_k (\mathbb{I}[t = 2020 + k]  \text{S}_{i})  + \gamma\mathbf{X}_{ist} + \lambda_s + \mu_g + \tau_t + \varepsilon_{isgt}
    \end{align}   


where $Y$ denotes a student’s standardized test scores, grade point average (normalized to have mean zero and standard deviation one at the school-grade-year level), passing rates and other educational outcomes. $S$ is an indicator variable taking the value one if the individual has siblings, henceforth is \textit{‘more exposed’}, $Post$ is an indicator variable taking the value one if the year is 2020 and over, to account for the beginning of school closures. $X$ is a set of controls that depending on the analysis can be sex, parental education, parents' age and baseline characteristics of the student and the household. I also include a set of school  ($\lambda$), grade  ($\mu$) and time ($\tau$) fixed effects.  The coefficient of interest is captured by $\beta$, which represents difference in achievement gaps for \textit{more exposed} versus \textit{less exposed} children, that is, between children with siblings and only children. I also use an event study specification for a similar analysis.

Even if I find a differential effect between both groups, one concern is that the effect captured reflects only the heterogeneity on another dimension correlated with family size such as previous academic performance and socio-economic status. In order to address this I take two different approaches. First, taking advantage of the large administrative data, in \hyperref[fig:main_result]{Figure \ref{fig:main_result}} I estimate the TWFE for different subsamples of the population and find that results are generally robust to all groups. Second, using additional information from standardized national examinations taken at different years to the same group of students, I estimate effects while controlling for baseline achievement and socioeconomic levels. Also, by using information from baseline surveys, I can perform more precise heterogeneity analysis that sheds some light into potential mechanisms. \hyperref[tab:twfe_gpa_baseline_survey_1_pairall]{Table \ref{tab:twfe_gpa_baseline_survey_1_pairall}}
 and \hyperref[tab:twfe_gpa_baseline_survey_2_pairall]{Table \ref{tab:twfe_gpa_baseline_survey_2_pairall}}
.

%This is an estimate specific to families with three or more children, and as demonstrated in Table 1 these families are somewhat different in their observable characteristics from families with two children—the minimal family size for spillovers to exist. Although we do not have a causal design at hand that would allow estimation of these effects in smaller families, we attempt to test for generalisability by re-estimating our models for families with three or more children weighted with characteristics of households with two children only. These results are qualitatively similar to our main findings.

%The key identifying assumption in a difference-in-differences approach is that there are no differential trends between the treatment and control group in the absence of the treatment. One concern is that, even before the third child is born, families with disabled third children may be trending in their risk of adverse outcomes in a different manner than are those without disabled third children. If the outcomes of children in families with disabled third children were already on a downward trajectory for reasons that are unrelated to the third-child’s disability, this could bias our results and lead us to overestimate the effect of sibling spillovers.15 We investigate this common trends assumption by studying, among others, birth weight and fiveminute Apgar scores as outcomes of the analyses. While the identifying assumption cannot be tested directly, we would interpret potential ‘effects’ on prior birth outcomes as an indication of confoundedness. 

I estimate the main regression equation (1) for the entire sample of first-borns as well as for several subgroups defined by school characteristics (rural, public, etc), student characteristics (sex, grade, etc), sibling characteristics (age gap, etc) and parent characteristics (mother's education, living with both parents, etc).


%%%%%%%%%%%%%%
% Results
%%%%%%%%%%%%%%
\section{Results}\label{sec:results}

The main results are shown in \hyperref[fig:main_result]{Figure \ref{fig:main_result}}. The event study estimates in \hyperref[fig:main_result_event]{Figure \ref{fig:main_result_event}} show clearly that previous to school closures, each of the \textit{more exposed} groups trends similarly compared to the \textit{less exposed} group of only children before school closures. It is after schools closed that I start seeing a break in the trend with larger effects for children with more siblings. After schools reopened, students with only one sibling return to their pre-pandemic levels relative to only children but those with two or more siblings show persistent effects.

\subsection{Learning Losses}

%Even though home production contributes more to reading than to mathematics performance \cite{black_recent_2010} \textcolor{green}{(check?)}, I can think of the larger losses in mathematics as a consequence of the school closures.

\subsubsection{GPA}

Given the administrative data has information on school grades on every subject, most of my analysis is based on standardized GPA at the school-grade-year level. In this way, by controlling for school grade and year fixed effects, the analysis is based on the relative differences between only children and children with siblings at the classroom level. Overall, I find significant negative effects in the reductions of GPA.  In \hyperref[fig:main_result_twfe]{Figure \ref{fig:main_result_twfe}} I show how these results are consistent for different subgroups of the population. Particularly interesting is how effects are significantly larger for primary school students than for secondary school students. Results are explored in more depth in \hyperref[fig:fig_appA]{Figure \ref{fig:fig_appA}}, \hyperref[fig:fig_appB]{Figure \ref{fig:fig_appB}}, \hyperref[fig:fig_appC]{Figure \ref{fig:fig_appC}}, and \hyperref[fig:fig_appD]{Figure \ref{fig:fig_appD}}.

\subsubsection{Standardized Exams}

In \hyperref[tab:twfe_ece]{Table \ref{tab:twfe_ece}}
I show significant effects on national standardized examinations taken in 2022 and 2023, when schools had already re-opened. The effect of these effects is big and significant, especially on lower grades, consistent with the analysis by GPA.

\subsection{Educational Trajectories}

%\subsubsection{Grade Progression}

%\textcolor{green}{Pending}

%\subsubsection{Graduation and Higher Education}

%\textcolor{green}{Pending}

\subsubsection{Expectations over educational attainment}

In \hyperref[tab:twfe_ece]{Table \ref{tab:twfe_ece}}
I show significant reductions in parental expectations over their children reaching a 4 year college degree. These effects are more clearly present in 4th grade students, were losses in academic achievement are the largest.

%%%%%%%%%%%%%%
% Robustness and Mechanisms
%%%%%%%%%%%%%%
\section{Mechanisms}\label{sec:mechanisms}

I have found a general pattern, present in many different segments of the population in Peru, that students with siblings exhibited larger learning losses during school closures when compared to only children. Even more, the size of the learning loss is increasing with the number of siblings. There are a number of plausible explanations for why this pattern would exist. First, to address any potential birth order effects, all the estimates that have been discussed compare first-born children. Second, siblings could be having a direct effect either by having to share common resources like computers or study rooms or by being a distraction to each other. Third, siblings could be affecting each other indirectly through the dilution of parental time available to each children. Fourth, bigger families could experience bigger income shocks due to the general lockdown and potential job loss. This could then have a negative effect on those families.


\subsection{Birth Order}

To separate birth order from family size effects, I have considered only estimates using first-born children. Results are however consistent when considering other children in the \textit{more exposed} group. This allows for different potential analysis, such as the effect by age of the oldest sibling although this may increase concerns about the comparison with only children given the increasing differences between both samples.

\subsection{Sibling disruptions}

If siblings were being detrimental for learning through disruptive behaviour, I would expect this to be more prevalent when siblings are close in age gap. However, in \hyperref[fig:fig_appC]{Figure \ref{fig:fig_appC}} I show that results are similar when the students with siblings considered for the estimation are those with siblings close relative in age (0-2 years of age gap) or with large age gaps (6 years or more). Additionally, results do not seem to be caused by siblings fighting for material resources either. In panel D and E of \hyperref[tab:twfe_gpa_baseline_survey_1_pairall]{Table \ref{tab:twfe_gpa_baseline_survey_1_pairall}} I show that the negative effects are present even in households with neither a computer or internet to access remote education easily.

%\subsection{Socio economic and human capital constraints in the household}




\subsection{Parental time and investment}

I find that the most likely mechanism driving these results is parental time and investment in their children's education. Given the reduced role of schools and teachers in the education production function through school closures, parents role becomes more prevalent and relevant. But they face constraints and can only allocate so much of their time to their children, and less so the more children they have.

Consistent with this hypothesis, in \hyperref[fig:fig_appB]{Figure \ref{fig:fig_appB}} I see results are larger for lower grades, when parental investments are more important. Additionally, in \hyperref[tab:twfe_gpa_baseline_survey_1_pairall]{Table \ref{tab:twfe_gpa_baseline_survey_1_pairall}} I show that the negative effects are driven mostly from high performing students and students whose parents have higher expectations from parental education. Even more, when looking at students whose baseline achievement was in the bottom quartile, I see no negative effect in the population of students with only one sibling when compared to only children. There are two potential explanations for this. On one hand, families with higher ability students or those that have higher expectations, tend to invest more of their time in the education of their children and hence these causes a dilution effect when there is more than one children while there is no effect in families who do not spend as much time. On the other hand, this could be suggestive of potential compensating effects, that is, parents  unequally dividing their time focusing more on students who are doing worse. The latter would point in an opposite direction of what was found in recent research by \cite{giannola_parental_2024}.

To further this analysis, I explore a different strategy by exploiting school starting ages (SSA) in Peru. In \hyperref[tab:rd_summ_1_m_a_365]{Table \ref{tab:rd_summ_1_m_a_365}} I show how delaying school has a negative effect in the older sibling when schools operate normally, potentially showing the effects of increased childcare of having a younger sibling stay at home. During school closures, the student born before the cutoff would have to also stay at home rather than go to school. Even more, the potential spillover of having a younger sibling stay at home is likely larger during this period given the increased importance of parental investment. This results are also somewhat significant when looking at standardized test score in \hyperref[tab:rd_ece_index_365]{Table \ref{tab:rd_ece_index_365}}. But does this mean that having a younger sibling stay at home reduces parental investment in older siblings? Column 5 of \hyperref[tab:rd_ece_index_365]{Table \ref{tab:rd_ece_index_365}} shows a reduction of 0.035 standard deviations in an parental time in education investment index based on how much they help their child with school work, study, homework, etc.

However, there are some results that are not consistent with this. I would expect single parents to be even more constrained in their time. However in \hyperref[fig:fig_appD]{Figure \ref{fig:fig_appD}} I show that results are similar for students that live with both parents or with only one parent.
%What about lives with mother only?

\subsection{Income}

There is no information on income but rather a socio-economic index based on household characteristics. In \hyperref[tab:twfe_ece]{Table \ref{tab:twfe_ece}} I show results on 2022 and 2023 socio-economic status index based on household characteristics from the survey. There are positive small but significant differences in some cases, although pointing in a direction opposite to negative income shocks as a mechanisms. That is, the socioeconomic status of larger families has slightly improved relative to that of only children. One caveat is that this index is more rigid than income and families could be potentially experiencing income shocks without an immediate effect in the socio-economic index, which is based on house materials, access to services and material belongings.

Additionally, I explore heterogeneous results on GPA by looking at households based on their baseline socio-economic status. In \hyperref[tab:twfe_gpa_baseline_survey_1_pairall]{Table \ref{tab:twfe_gpa_baseline_survey_1_pairall}} I show that results are larger in households from the top quartile of SES, consistent with them being the ones who often spend more time in education of their children. Still, results in the bottom quartile of SES are still significant and large, although present only when students have 2 or 3 siblings.




%%%%%%%%%%%%%%
% Policy Discussion
%%%%%%%%%%%%%%
%\section{Policy Discussion}\label{sec:policy}

%%%%%%%%%%%%%%
%Conclusions
%%%%%%%%%%%%%%
\section{Conclusions}\label{sec:conclusions}

This paper has found evidence of a so far overlooked issue regarding family structure and school closures: That larger families struggled more to fill the role left by schools, that the losses caused by this are persistent and that they are likely caused by parents being unable to substitute the role of teachers given time constraints that become more prevalent when having to attend multiple children.

Peru was one of the countries that was most affected by the pandemic, both in death rates and in restrictive measures taken by the government. This begs the question about how valid are these results in other contexts. Based on the change in PISA test scores from 2012 to 2022 and the severity of school closures, in \hyperref[fig:pisa]{Figure \ref{fig:pisa}} I see that most countries experienced a similar pattern of larger losses for children with sibligns and even more, I see these losses were larger in countries with longer school closures. This pattern occurs for both developing and developed countries.

%We argue that an older sibling’s enrollment in a higher quality college can provide for families information about postsecondary education that would otherwise be difficult or impossible to obtain. [draft US]




