
%%%%%%%%%%%%%%
% Introduction
%%%%%%%%%%%%%%

% 1. Motivation (1 Pargraph)


% Learning losses from school closures
%During the COVID-19 pandemic and school closures, many countries implemented some form of remote learning. Suddenly without the usual school inputs,  parents had a bigger role in the education of their children. This period caused significant learning losses, especially for more vulnerable populations (\cite{haelermans_inequality_2022}, \cite{jakubowski_global_2023}). Yet, with parents as a more relevant input in the education production function and remote schooling that required digital resources, there is no evidence on how the size of the household could have affected children's learning.


% Siblings did much worse (PISA score)

% Questions about family size being relevant?


% 2. Research question


% 6. Roadmap of paper.
The paper proceeds as follows. \hyperref[sec:empirical_strategy]{Section \ref{sec:empirical_strategy}} details the empirical approach. \hyperref[sec:data]{Section \ref{sec:data}}  describes the higher education system in Peru and linked administrative data. \hyperref[sec:results]{Section \ref{sec:results}} presents results. \hyperref[sec:mechanisms]{Section \ref{sec:mechanisms}} discusses potential mechanisms. \hyperref[sec:heterogeneity]{Section \ref{sec:heterogeneity}} discusses heterogeneity in results. \hyperref[sec:robustness]{Section \ref{sec:robustness}} shows robustness checks. \hyperref[sec:policy]{Section \ref{sec:policy}} discusses policy implications. \hyperref[sec:conclusions]{Section \ref{sec:conclusions}} concludes.

\hyperref[fig:pisa]{Figure \ref{fig:pisa}}

\hyperref[fig:fig2]{Figure \ref{fig:fig2}}


\hyperref[tab:twfe_ece]{Table \ref{tab:twfe_ece}}

\hyperref[tab:twfe_ece_survey_1]{Table \ref{tab:twfe_ece_survey_1}}

\hyperref[tab:twfe_ece_survey_2]{Table \ref{tab:twfe_ece_survey_2}}

\hyperref[tab:rd_ece_index_365]{Table \ref{tab:rd_ece_index_365}}

\hyperref[tab:rd_summ_K_m_a_365]{Table \ref{tab:rd_summ_K_m_a_365}}

\hyperref[tab:rd_summ_2_m_a_365]{Table \ref{tab:rd_summ_2_m_a_365}}


%%%%%%%%%%%%%%
% Empirical Strategy
%%%%%%%%%%%%%%

\section{Empirical Strategy}\label{sec:empirical_strategy}


%%%%%%%%%%%%%%
% Data
%%%%%%%%%%%%%%
\section{Data}\label{sec:data}

%%%%%%%%%%%%%%
% Results
%%%%%%%%%%%%%%
\section{Results}\label{sec:results}

\subsection{Learning Losses}

\subsubsection{GPA}

\subsubsection{Standardized Exams}

\subsection{Educational Trajectories}

\subsubsection{Grade Progression}

\subsubsection{Graduation and Higher Education}


%%%%%%%%%%%%%%
% Robustness and Mechanisms
%%%%%%%%%%%%%%
\section{Mechanisms}\label{sec:mechanisms}

Them main potential mechanisms are...

\subsection{Birth Order}

Results hold when working only with first births

\subsection{Socio economic and human capital constraints in the household}

Results hold when controlling for these

\subsection{Sibling disruptions}

Results hold when doing age gap.

\subsection{Parental time and investment}

Results by single/both parents

Results from RD of SSA

Results from \hyperref[tab:rd_summ_1_m_a_365]{Table \ref{tab:rd_summ_1_m_a_365}} show how delaying school has a negative effect in the older sibling. \footenote{a}


Results from birth of a sibling

\section{Heterogeneity}\label{sec:heterogeneity}

Surprisingly, results are consistent across...

\subsection{Reinforcement vs Compensation}

\section{Robustness}\label{sec:robustness}

%%%%%%%%%%%%%%
% Policy Discussion
%%%%%%%%%%%%%%
\section{Policy Discussion}\label{sec:policy}

%%%%%%%%%%%%%%
%Conclusions
%%%%%%%%%%%%%%
\section{Conclusions}\label{sec:conclusions}


%We argue that an older sibling’s enrollment in a higher quality college can provide for families information about postsecondary education that would otherwise be difficult or impossible to obtain. [draft US]




