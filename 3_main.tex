
%%%%%%%%%%%%%%
% Introduction
%%%%%%%%%%%%%%

%Motivation x1
%Research question x1
%Method/Unique Features x1-2
%Findings x1-2
%Mechanisms/Robustness checks x2
%Contributions to Literature x3
%Policy Relevance/Big Picture x1



% 1. Motivation (1 Pargraph)

The relationship between family size and quality of education given to each children has been long studied in economics. Models from \cite{becker_child_1976} and ... suggested a quantity-quality tradeoff although many empirical evidence has suggested otherwise (\cite{black_more_2005}). However, when faced with unexpected shocks to family size, parental time and financial resources, the quality of education that children receive might be affected, especially if they are exposed to it during longer periods or crucial developmental stages (\textcolor{green}{cite?}). These suggest that parents may be able to adapt to planned increases in family size (and childcare responsibilities) but if they are incapable of that, the quality of education that children receive is affected. Given that one of the main mechanisms through which this operate is parental time a financial constraints, we can expect it to have a significant relevance in a context that increasingly puts this to the test.

School closures occurred with the COVID-19 pandemic, in some places for weeks and in other for more than a year and with them, the household became a bigger actor in the education production function of the children. There are reported learning losses that persist even after schools reopened, and that were particularly strong for more vulnerable populations such as low income students and those without access to internet (\cite{haelermans_inequality_2022}, \cite{jakubowski_global_2023}). There are no negative effects of family size reported among these, however, from the literature on ... we can expect it to play a major role, especially when...


% Learning losses from school closures
%During the COVID-19 pandemic and school closures, many countries implemented some form of remote learning. Suddenly without the usual school inputs,  parents had a bigger role in the education of their children. This period caused significant learning losses, especially for more vulnerable populations (\cite{haelermans_inequality_2022}, \cite{jakubowski_global_2023}). Yet, with parents as a more relevant input in the education production function and remote schooling that required digital resources, there is no evidence on how the size of the household could have affected children's learning.


% Siblings did much worse (PISA score)

% Questions about family size being relevant?


%2. Research question x1

This paper makes use of administrative-level data from Peru to estimate the effect of a shock in childcare and education through nationwide lockdowns and school closures in the outcomes of children in the family. I use family size to study the exposure to this shock; specifically, I examine how different are the attainment and learning losses in children that have siblings compared to children who do not, measured from 1st to 11th grade, when they graduate from high school. Even though our estimation is based on a extreme situation, we believe it allows us to learn about how families reallocate resources when time and attention from parents is suddenly more required, and how bigger families may be less able to adapt under such circumstances. Sibling spillovers might occur through direct effects (mentoring or competing for resources like computers) or indirect effects through parental resource allocation, as increased childcare reduces time for educational activities. With reduced participation from schools and teachers in students’ education and children being at home, parental investments in education become more important but more constrained due to increased childcare responsibilities; hence, these sibling spillovers may be exacerbated. 


%3. Method/Unique Features x1-2
To address the identification issue, we compare families with multiple children, that arguably experience a larger shock in childcare requirements, to those with only one child. Because the number of children is not randomly distributed across families, we look at the differences across time.  Even though these two groups are different, they have parallel trends before school closures and in that regard, our estimate is how differentially affected they have been as a consequence of the difference in exposure to shocks in parental time requirements. I am thus comparing outcomes of the first-born in families with multiple children (more affected) and those where they are the only children (less affected) before and after school closures. In doing so, I eliminate biases associated with unobserved time-invariant family characteristics.

While this approach allows us to estimate the effects of relative increases in childcare, we are unable to estimate the total effect of increases in childcare due to school closures. Given that families with only one children also experience significant increases in the parental time required to care and educate their child at home, our estimate is just how much more are families with multiple children affected by this. However, we can see the increases with the number of siblings as potential marginal effects of... Another limitation of our research design is that it relies on the assumption that trends....

Additionally, to address concerns about potential issues when comparing students with different amount of siblings. I also leverage two other potential sources of variation. First, I explore school starting age cutoffs and sibling spillovers. During normal times, having a younger sibling born before the cutoff meant having them go to school early and release parents from some of the care required. During school closures, that same younger sibling would have to stay in the house, potentially diluting parental resources and affecting their other siblings. Second, I compare families with one child that have their second child born just before or after the school closures....



\textcolor{red}{From Sandy's: While this approach allows us to estimate the effects of relative exposure to a disabled sibling, we are unable to estimate the total effect of exposure to a disabled sibling. \textbf{Because families with a disabled child may be different in unobserved ways from other families, we are unable to distinguish whether differences in outcomes are the result of exposure to a disabled child or of differences in unobservables across these families.}}

\textcolor{blue}{A key underlying assumption of our identification strategy is that the effect of birth order does not vary based on the presence of a disabled third sibling, other than through the differential exposure to the disabled third sibling. While we cannot test this directly, we do conduct a number of exercises to suggest that this might be a reasonable assumption.

Because our research strategy is based on differential exposure to disabled younger siblings in early life, it is particularly important to observe disabilities that are noticeable to families early. We make use of administrative data that reveal when children first obtain services for their disabilities, and we concentrate on disabilities that are first identified in or before kindergarten. It is harder to separate the spillover effect from the common-shock effect for disabilities first diagnosed in the elementary school years because disabilities diagnosed later are more likely to reflect unobserved family factors rather than a singular health or physical shock. Furthermore, as an alternative to measuring the effects of disabilities that appear in early childhood, we also consider the spillover effects of exposure to a third-born sibling with health problems at birth, such as congenital anomalies, abnormal conditions, low birth weight, prematurity or very poor perinatal health. We find similar results regardless of whether we study the spillover effects of the third-born child’s disability as recorded in early childhood or of the child’s poor birth conditions.
.
We estimate these relationships using data from two data sources, drawn from two distinct locations: Florida and Denmark. In both cases, we make use of matched administrative data sets where we merge full birth cohorts with schooling and medical data. The large sample sizes allow us to generate good statistical power while focusing on families with at least three children, a subset of which has a third child with disabilities. The data also allow us to follow children for whom we know birth outcomes and family structure into early adolescence to measure cognitive development.
.
The Florida data include birth records for every birth in the state between 1994 and 2002, merged with school records for all children who attended public schools within the state. The key outcome we measure in Florida is the state’s criterion-referenced end-of-year standardised test, the Florida Comprehensive Assessment Test (FCAT), administered to children in grades 3–8 over the relevant time period. (For more detail on these data, see Figlio et al., 2013 and Figlio et al., 2014.)
.
The Danish data include birth records for every birth in the country between 1990 and 2001, merged with medical patient register and school records. Our main outcome of interest in Denmark is the cohort-standardised grade from the 9th grade exit exam (GPA). These exam results are based on assessments by the student’s teacher and by an external reviewer. While these exam scores do not capture exactly the same features as the Florida exams, they are widely interpreted to represent similar types of outcomes in the extant literature.
.
Including data from two very different contexts provides a number of advantages. For one, the United States and Denmark have very different educational systems, healthcare systems and social welfare systems. To the degree to which we find comparable results in these two quite different environments, it further increases our confidence that the estimates are externally valid and more generally reflect spillovers and dynamics within the family, as opposed to something that is very context-specific. Second, disability is measured differently in the two contexts—on the education record in Florida and on the medical record in Denmark. This fact also aids in establishing the degree to which our findings are externally valid. Furthermore, having data from both settings provides greater opportunities to explore the mechanisms through which these findings might be operating because the two countries have quite different population characteristics and different data strengths. Thus, between the two sites, we are able to paint a more complete picture about the nature of the sibling spillovers that we seek to investigate.}


%4. Findings x1-2


%5. Mechanisms/Robustness checks x2
I explore different potential mechanisms behind these effects. First, to avoid confounding birth order effects with family size effects, I focus on the first-born children both when comparing only children with children with siblings. Second, siblings could be directly affecting each other. We could expect closely spaced siblings to be more disruptive to each other when sharing a study environment, however, results are similar when the age gap between siblings is larger. Also, one could expect siblings fighting for a computer during remote schooling, however, results are similar in households without PC and internet than those who had them. Third, siblings might have indirect effects, mainly through their effect on parental time: not being able to leave them in childcare, pre-K, or school, parents now need to not only take up on increased childcare responsibilities, an increase that is bigger the more children they have, but also on schooling ones. Most evidence points in this direction: (i) results are larger in children from elementary school, where parental time investments are more important for development, (ii) leveraging school starting ages (SSA) I find that having a child stay in the household rather than going to school decreases older siblings' scores and parental investment (iii) Results when focusing on parents by level of education and living situation .... (\textcolor{green}{Check also with ECE}. Fourth, bigger families might experience a larger income shock from lockdowns which may as a consequence affect children from those families more. However, we do not find significant effects in changes in socioeconomic status of those families. I also find robust effects in both low and high income families, and evene larger ones in families from the top quartile of SES, which are more equipped to deal with income losses and at the same time spend more time with their children (\textcolor{green}{reference})

%6. Contributions to Literature x3

%% Literature branch #1: Sibling spillovers in education trajectories

%: Our results suggest that the effect of family size depends on the type of family-size intervention and that there are no important negative effects of expected increases in family size. However, unexpected shocks to family size resulting from twin births have negative effects on the IQ scores of existing children. 


%: The paper finds evidence that, relative to the first born, the second child in a family is differentially affected when the third child is disabled.


My results contribute to X strands of research. First, it relates to the literature of family size that applying their findings the negative effect of unexpected changes into another context that requires much more parental involvement that would otherwise be covered by schools ... \cite{black_small_2010}.... Second, it relates to the literature of health shocks ...\cite{black_sibling_2021}... Third, it relates to the literature of COVID effects on education.

%% Literature branch #2: Information frictions, expectations... 
%Second, the results I find on mechanisms contribute to the literature on 




%7. Policy Relevance/Big Picture x1
Policy relevance.. (i) Effects are persistent: opportunity for interventions on affected population. (ii) Effects are widespread across all segments of the population but also across countries (\hyperref[fig:pisa]{Figure \ref{fig:pisa}}), especially those most affected by school closures.



% 6. Roadmap of paper.
The paper proceeds as follows. \hyperref[sec:empirical_strategy]{Section \ref{sec:empirical_strategy}} details the empirical approach. \hyperref[sec:data]{Section \ref{sec:data}}  describes the higher education system in Peru and linked administrative data. \hyperref[sec:results]{Section \ref{sec:results}} presents results. \hyperref[sec:mechanisms]{Section \ref{sec:mechanisms}} discusses potential mechanisms. \hyperref[sec:heterogeneity]{Section \ref{sec:heterogeneity}} discusses heterogeneity in results. \hyperref[sec:robustness]{Section \ref{sec:robustness}} shows robustness checks. \hyperref[sec:policy]{Section \ref{sec:policy}} discusses policy implications. \hyperref[sec:conclusions]{Section \ref{sec:conclusions}} concludes.

\hyperref[fig:pisa]{Figure \ref{fig:pisa}}

\hyperref[fig:fig2]{Figure \ref{fig:fig2}}


\hyperref[tab:twfe_ece]{Table \ref{tab:twfe_ece}}

\hyperref[tab:twfe_gpa_baseline_survey_1_pairall]{Table \ref{tab:twfe_gpa_baseline_survey_1_pairall}}

\hyperref[tab:twfe_gpa_baseline_survey_2_pairall]{Table \ref{tab:twfe_gpa_baseline_survey_2_pairall}}

\hyperref[tab:rd_ece_index_365]{Table \ref{tab:rd_ece_index_365}}

\hyperref[tab:rd_summ_K_m_a_365]{Table \ref{tab:rd_summ_K_m_a_365}}

\hyperref[tab:rd_summ_2_m_a_365]{Table \ref{tab:rd_summ_2_m_a_365}}

\hyperref[fig:fig_appA]{Figure \ref{fig:fig_appA}}

\hyperref[fig:fig_appB]{Figure \ref{fig:fig_appB}}

\hyperref[fig:fig_appC]{Figure \ref{fig:fig_appC}}

\hyperref[fig:fig_appD]{Figure \ref{fig:fig_appD}}


\hyperref[tab:twfe_gpa_baseline_survey_1_pair1]{Table \ref{tab:twfe_gpa_baseline_survey_1_pair1}}

\hyperref[tab:twfe_gpa_baseline_survey_2_pair1]{Table \ref{tab:twfe_gpa_baseline_survey_2_pair1}}


\hyperref[tab:twfe_gpa_baseline_survey_1_pair2]{Table \ref{tab:twfe_gpa_baseline_survey_1_pair2}}

\hyperref[tab:twfe_gpa_baseline_survey_2_pair2]{Table \ref{tab:twfe_gpa_baseline_survey_2_pair2}}


\hyperref[tab:twfe_gpa_baseline_survey_1_pair3]{Table \ref{tab:twfe_gpa_baseline_survey_1_pair3}}

\hyperref[tab:twfe_gpa_baseline_survey_2_pair3]{Table \ref{tab:twfe_gpa_baseline_survey_2_pair3}}


\hyperref[tab:twfe_gpa_baseline_survey_1_pair4]{Table \ref{tab:twfe_gpa_baseline_survey_1_pair4}}

\hyperref[tab:twfe_gpa_baseline_survey_2_pair4]{Table \ref{tab:twfe_gpa_baseline_survey_2_pair4}}



%%%%%%%%%%%%%%
% Empirical Strategy
%%%%%%%%%%%%%%

\section{Empirical Strategy}\label{sec:empirical_strategy}

Our research approach is to carry out a simple difference-in-differences design: we compare second- versus first-born children, in families with disabled and non-disabled third children. The basic idea underlying the research design is that, because we observe outcomes at a fixed common age, second-born children have spent a larger share of their lives to that point exposed to a disabled sibling than first-borns have. However, first- and second-born children may have different outcomes because of the direct effects of birth order. We therefore subtract off the first- versus second-born difference measured among families that have non-disabled third-born children as a way of separating the birth-order effect from the effect of differences in exposure to the third-born sibling. 

Put differently, the comparisons in Table 1 make it clear that families with disabled children are different in some ways from families with non-disabled children. This means that comparisons between children who have disabled siblings and children who do not have disabled siblings may be confounded by differences in unobservables. Our research design accounts for those differences across families by making comparisons within families, across first- and second-born siblings, who have different amounts of exposure to the third-born siblings by the time they reach any given age. 

Our main regression equation is the following:

    \begin{align}
    Y_{isgt} &= \alpha + \delta_1 \text{Post}_{it} + \delta_2 \text{S}_{i}   + \boldsymbol{\beta} \text{Post}_{it} \text{S}_{i}  + \gamma\mathbf{X}_{ist} + \lambda_s + \mu_g + \tau_t + \varepsilon_{isgt}
    \end{align}

     \begin{align}
    Y_{isgt} &= \alpha + \delta_1 \text{Post}_{it} + \delta_2 \text{S}_{i}   + \sum_{k=-5}^{-2} \delta_k (\mathbb{I}[t = 2020+k] \text{S}_{i}) \nonumber \\
    & \quad + \sum_{k=0}^{4} \beta_k (\mathbb{I}[t = 2020 + k]  \text{S}_{i})  + \gamma\mathbf{X}_{ist} + \lambda_s + \mu_g + \tau_t + \varepsilon_{isgt}
    \end{align}   


$M$ is an indicator variable taking the value one if the individual has siblings, henceforth is ‘more exposed’.
    
where Y denotes a student’s standardised test scores or grade point average (normalised to have mean zero and standard deviation one), D is an indicator variable taking the value one if the third child is observed disabled and zero otherwise, and M is an indicator variable taking the value one if the individual is second-born, henceforth ‘more exposed’, and zero if the individual is firstborn, or ‘less exposed’. The indicator variable for third-child disability (D) is not identified in this fixed effects model because it is constant across the first two births. The vector of covariates, X, includes indicator variables for year and month of birth as well as child gender. The index i denotes individuals and f denotes family, and αf is the family fixed effect. Our main parameter of interest is β2, which represents the difference in achievement gaps for more exposed versus less exposed siblings in families with and without a disabled third child, in a model in which family fixed effects net out any time invariant differences between families with and without a disabled third child. This is an estimate specific to families with three or more children, and as demonstrated in Table 1 these families are somewhat different in their observable characteristics from families with two children—the minimal family size for spillovers to exist. Although we do not have a causal design at hand that would allow estimation of these effects in smaller families, we attempt to test for generalisability by re-estimating our models for families with three or more children weighted with characteristics of households with two children only. These results are qualitatively similar to our main findings.

The key identifying assumption in a difference-in-differences approach is that there are no differential trends between the treatment and control group in the absence of the treatment. One concern is that, even before the third child is born, families with disabled third children may be trending in their risk of adverse outcomes in a different manner than are those without disabled third children. If the outcomes of children in families with disabled third children were already on a downward trajectory for reasons that are unrelated to the third-child’s disability, this could bias our results and lead us to overestimate the effect of sibling spillovers.15 We investigate this common trends assumption by studying, among others, birth weight and fiveminute Apgar scores as outcomes of the analyses. While the identifying assumption cannot be tested directly, we would interpret potential ‘effects’ on prior birth outcomes as an indication of confoundedness. 

We estimate the main regression equation (1) for the entire sample as well as for several subgroups defined by, e.g., mother’s education, type of disability or its severity.

%%%%%%%%%%%%%%
% Data
%%%%%%%%%%%%%%
\section{Data}\label{sec:data}

\subsection{SIAGIE}

\subsection{ECE}

\subsection{Sibling Identification}

\subsection{Descriptive statistics}

\textcolor{green}{Table of descriptive statistics by family size}
\textcolor{green}{Figure of trends. Use actual GPA instead of standardized?}

%%%%%%%%%%%%%%
% Results
%%%%%%%%%%%%%%
\section{Results}\label{sec:results}

\subsection{Learning Losses}

Even though home production contributes more to reading than to mathematics performance \cite{black_recent_2010} \textcolor{green}{(check?)}, we can think of the larger losses in mathematics as a consequence of the school closures.

\subsubsection{GPA}

\subsubsection{Standardized Exams}

\subsection{Educational Trajectories}

\subsubsection{Grade Progression}

\subsubsection{Graduation and Higher Education}

\subsection{Expectations over educational attainment}

Using ECE survey from before and after covid we perform simple TWFE estimations. In \textcolor{green}{Table...} we see a reduction in parental expectations for attaining a 4-year college degree and an increase of those who believe the maximum level of education will be high-school.


%%%%%%%%%%%%%%
% Robustness and Mechanisms
%%%%%%%%%%%%%%
\section{Mechanisms}\label{sec:mechanisms}

We have found a general pattern, present in both Denmark and Florida, that a disabled third sibling differentially affects the second-born child relative to the oldest child. There are a number of plausible explanations for why this pattern would exist. For one, siblings who are more closely spaced in age are probably more likely to spend time together in mutual activities, so there are more opportunities for direct spillovers between the third-born sibling and the second-born sibling than between the third-born sibling and the first-born sibling. That the results are only present for physical disabilities which are diagnosed very early in childhood suggests that the direct channel may not be the primary driver of the sibling spillover we estimate. Were it due to direct effects stemming from engagement with the disabled sibling, we would have expected the effects to be present across a range of disabilities and ages.

\subsection{Birth Order}

Results hold when working only with first births

\subsection{Sibling disruptions}

Results hold when doing age gap.

\subsection{Socio economic and human capital constraints in the household}

Results hold when controlling for level of education, internet and pc.

We also look into the number of rooms\

\subsection{Parental time and investment}

Results by single/both parents. Other adults in the household.

Results from RD of SSA

Results from \hyperref[tab:rd_summ_1_m_a_365]{Table \ref{tab:rd_summ_1_m_a_365}} show how delaying school has a negative effect in the older sibling. \footenote{a}

\subsection{Income}

Results from socio-economic status show that... 
This is highly correlated with income... (show)
However, it is certainly not a malleable, but it is evidence that there is not differential loss in income driving the results.

Perhaps show that there are losses of socio-economic status overall? Can we do this with household surveys or with the ECE surveys themselves?

\section{Heterogeneity}\label{sec:heterogeneity}

Surprisingly, results are consistent across...

\subsection{Reinforcement vs Compensation}

\section{Robustness}\label{sec:robustness}

\subsection{Potential sample selection}

Discuss how siblings are observed. We only see observations as long as they are enrolled in data. Most children are enrolled in primary school \textcolor{green}{(show data from household surveys)}. However, most children are not enrolled until 3-4 years old. Since our last year of administrative data is 2024, this means that some families with 2 children, one of which was born after 2021, might be observed as only childs. To address any concerns from this selection I do the following:

\begin{itemize}
    \item These families are not particularly different from others? (this true)?
    \item I use data up to 2023 and 2022 to define families (and potentially imposing a similar bias to earlier years) and still see results change around COVID. \textcolor{green}{Pending Analyis}
\end{itemize}


\subsection{Siblings in same vs different schools}

\subsection{Different definitions of siblings: father, both, caretaker}

\subsection{Younger and middle child}

\subsection{Unadjusted scores}

%%%%%%%%%%%%%%
% Policy Discussion
%%%%%%%%%%%%%%
\section{Policy Discussion}\label{sec:policy}

%%%%%%%%%%%%%%
%Conclusions
%%%%%%%%%%%%%%
\section{Conclusions}\label{sec:conclusions}


%We argue that an older sibling’s enrollment in a higher quality college can provide for families information about postsecondary education that would otherwise be difficult or impossible to obtain. [draft US]




