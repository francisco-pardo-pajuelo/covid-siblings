
%%%%%%%%%%%%%%
% Introduction
%%%%%%%%%%%%%%

% 1. Motivation (1 Pargraph)


% Learning losses from school closures
During the COVID-19 pandemic and school closures, many countries implemented some form of remote learning. Suddenly without the usual school inputs,  parents had a bigger role in the education of their children. This period caused significant learning losses, especially for more vulnerable populations (\cite{haelermans_inequality_2022}, \cite{jakubowski_global_2023}). Yet, with parents as a more relevant input in the education production function and remote schooling that required digital resources, there is no evidence on how the size of the household could have affected children's learning.

% Siblings did much worse (PISA score)

% Questions about family size being relevant?


% 2. Research question


% 6. Roadmap of paper.
%The paper proceeds as follows. \hyperref[sec:empirical_strategy]{Section \ref{sec:empirical_strategy}} details the empirical approach. \hyperref[sec:data]{Section \ref{sec:data}}  describes the higher education system in Peru and linked administrative data. \hyperref[sec:results]{Section \ref{sec:results}} presents results. \hyperref[sec:robustness_mechanisms]{Section \ref{sec:robustness_mechanisms}} discusses potential mechanisms. \hyperref[sec:conclusions]{Section \ref{sec:conclusions}} concludes.




%%%%%%%%%%%%%%
% Empirical Strategy
%%%%%%%%%%%%%%

%\section{Empirical Strategy}\label{sec:empirical_strategy}

%I do fuzzy regression discontinuity using college-major-semester (cell) cutoffs. (i) I estimate the likely cutoff for each cell, (ii) stack all applications, (iii) estimate an RD model with cell fixed effects. This is done with for the oldest sibling, with effects estimated on their younger siblings (after the application). This is done for all public universities in the country from 2017-2023.

%%%%%%%%%%%%%%
% Data
%%%%%%%%%%%%%%
%\section{Data}\label{sec:data}

%%%%%%%%%%%%%%
% Results
%%%%%%%%%%%%%%
%\section{Results}\label{sec:results}

% First stage













% Why does getting in matter? 
% Actual change in decisions and outcomes



     %The first stage is quite big. Being above the cutoff means a 70\% increased chance of admission and 50\% of enrollment to the applied cutoff. Being above the cutoff also increases enrollment in ANY university EVER by 18\%. \hyperref[fig:first_stage]{Figure \ref{fig:first_stage}}. Results for the full sample are as follow:

%%%%%%%%%%%%%%
% Robustness and Mechanisms
%%%%%%%%%%%%%%
%\section{Robustness and Mechanisms}\label{sec:robustness_mechanisms}


%%%%%%%%%%%%%%
% Policy Discussion
%%%%%%%%%%%%%%
%\section{Policy Discussion}\label{sec:policy}

%%%%%%%%%%%%%%
%Conclusions
%%%%%%%%%%%%%%
%\section{Conclusions}\label{sec:conclusions}


%We argue that an older sibling’s enrollment in a higher quality college can provide for families information about postsecondary education that would otherwise be difficult or impossible to obtain. [draft US]




