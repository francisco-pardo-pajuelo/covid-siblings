
\begin{abstract}

This paper examines how family structure mediates educational outcomes when unexpected shocks dramatically increase parental time requirements for children's learning. Using administrative data from Peru, I employ a difference-in-differences strategy comparing children with siblings to only children before, during, and after school closures caused by Covid-19. Students with siblings experienced significantly larger learning losses, with effects intensifying by number of siblings. These differential impacts persist after schools reopened and appear consistently across diverse subpopulations. Evidence points to parental time constraints as the primary mechanism. Effects are largest during primary education where parental investment matters most and in families with higher socio-economic resources which tend to spend more time with their children. Additionally, using school enrollment age cutoffs, I show that having younger siblings at home rather than at school significantly reduces older siblings' academic performance and measured parental investment. These effects extend beyond academic performance. Parents of students with siblings became systematically more pessimistic about their children's educational prospects. These findings reveal fundamental insights about family resource allocation under stress. When external education support disappears, the dilution of parental time across multiple children creates substantial disadvantages in larger families. %This has important implications for understanding how families adapt to crises and why larger families may be systematically vulnerable when parental time suddenly becomes the critical input in education production.

\\
\textit{JEL Codes: I21, I24, D13}
\end{abstract}

%   I2 Education and Research Institutions
    % 	I20 	General 
    %   I21 	Analysis of Education
    %	I23 	Higher Education • Research Institutions
    %	I24 	Education and Inequality
    %	I26 	Returns to Education 
    %   I28 	Government Policy     
    
%   D1 	Household Behavior and Family Economics 
    % 	D13 	Household Production and Intrahousehold Allocation 
    %   D19  other

%   J2 	Demand and Supply of Labor 
    %   J24 	Human Capital • Skills • Occupational Choice • Labor Productivity 

%   O1 	Economic Development 
    %   O15 	Human Resources • Human Development • Income Distribution • Migration 

%   R2 	Household Analysis  
    % 	R23 	Regional Migration • Regional Labor Markets • Population • Neighborhood Characteristics 





