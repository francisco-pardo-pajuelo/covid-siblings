
%%%%%%%%%%%%%%%%%%%%%%%%%%%%%%%%%%%%%%%%%%%%%%%%%%%%%%
%YOU HAVE ALREADY MADE A BACKUP. EDIT THIS FREELY
%%%%%%%%%%%%%%%%%%%%%%%%%%%%%%%%%%%%%%%%%%%%%%%%%%%%%%


\textbf{Research Question:} \textcolor{blue}{\textbf{`Do older siblings college experience affect their younger siblings aspirations, effort and actions over higher education?'}}


\textbf{Literature}
\begin{itemize}
    \item  Effects are strong on the choices they make, mainly following their older siblings to the same college. Less clear results on overall applications and enrollment. Null results on effort during high school.
    \item Some evidence of stronger effects when siblings are more similar (in age or sex). Mixed results and no clarity on mechanisms. \footnote{}\footnotetext{\cite{dustan_family_2018}, \cite{joensen_spillovers_2018}, \cite{altmejd_o_2021}, \cite{aguirre_walking_2021}, \cite{de_gendre_class_2021}, \cite{avdeev_spillovers_2024}, \cite{dahl_intergenerational_2024}}
\end{itemize}

\textbf{Motivation:}
\begin{itemize}
    \item Peru has decentralized applications and exams. This contrasts with systems like Chile, where there is a centralized application system (choice set) with a standardized exam for all applications. This creates bigger information barriers and costs for applications that may cause larger spillover effects. It also creates methodological challenges that a centralized system does not.
    \item I have school administrative data with exams in 2nd, 4th and 8th grade and survey data to students and parents on behavior, beliefs, expectations, etc. This allows for unexplored effects in early childhood and in mechanisms. In terms of outcomes, we can explore: 8th grade student's aspirations of going to college, 8th grade exams (earlier than literature) and full school grade progression.
\end{itemize}

\textbf{Strategy:} I do fuzzy regression discontinuity using college-major-semester (cell) cutoffs. (i) I estimate the likely cutoff for each cell, (ii) stack all applications, (iii) estimate an RD model with cell fixed effects. This is done with for the oldest sibling, with effects estimated on their younger siblings (after the application). This is done for all public universities in the country from 2017-2023.

\textbf{Results:}
     The
\newpage




%\section{Introduction}
%\label{sec:intro}


%\footnote{}\footnotetext{example footnote}

%medication/treatment  (\cite{sokol_impact_2005})
%\section{Literature review and contribution}
%\label{sec:literature}


%\section{Context}
%\label{sec:context}


%\section{Data}
%\label{sec:data}

%\section{Empirical Strategy}
%\label{sec:empirical}


%General RD
%\begin{align*}
%Y_{it}=\alpha_{i} + \beta {ABOVE} + \gamma  f(Age) + \delta X_{it} + \epsilon_{it} \numberthis \label{eq_main}
%\end{align*}



%\section{Mechanisms}
%\label{sec:mechanisms}


%\section{Heterogenity}
%\label{sec:heterogeneity}

%\section{Conclusion}
%\label{sec:conclusion}

%\subsection{Future Research}
%\label{sec:future}



%%%%%%%%%%%%%%%%%%%%%%%%%%%%%%%%%%%%%%
% Figures and Tables 
%%%%%%%%%%%%%%%%%%%%%%%%%%%%%%%%%%%%%%

%:::::::::::::: FIGURES


%\input{./figures/figure1.tex} 


%\clearpage

%:::::::::::::: TABLES


%\input{./tables/mechanisms.tex}


