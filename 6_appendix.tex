\clearpage
%\newpage

\setcounter{figure}{0}
\renewcommand\thefigure{A.\arabic{figure}}    

\setcounter{table}{0}
\renewcommand{\thetable}{A.\arabic{table}}
\setcounter{subsection}{0}

\begin{center}
\huge
\textbf{Appendix: NOT FOR PUBLICATION}
\normalsize
\end{center}


\section*{Appendix A: Additional Tables and Figures} \label{sec:appa}
\newpage



\begin{figure}[htbp]
         \centering
        \includegraphics[width=\textwidth]{./FIGURES/Descriptive/raw_grades_pass_math_siblings.pdf}
        \caption{\% of students with an A in Mathematics for each grade 1st-1th}
        \label{fig:trend_pass_grades}
\end{figure}

\begin{figure}[htbp]
         \centering
        \includegraphics[width=\textwidth]{./FIGURES/Descriptive/raw_grades_std_gpa_m_adj_siblings.pdf}
        \caption{Average GPA standardized within school-grade-year for each grade 1st-1th}
        \label{fig:trend_gpa_grades}
\end{figure}



\begin{figure}[htbp]
    \centering
    
        \includegraphics[width=\textwidth]{./FIGURES/TWFE/covid_twfe_A_bysibs_elm_all_gpa_m_adj_Tsiblings_Soldest_4.pdf}
        \caption{Change in gap between children with siblings and only childs}
        \label{fig:fig_appA}

\end{figure}

\begin{figure}[htbp]
    \centering
    
        \includegraphics[width=\textwidth]{./FIGURES/TWFE/covid_twfe_B_bysibs_elm_all_gpa_m_adj_Tsiblings_Soldest_4.pdf}
        \caption{Change in gap between children with siblings and only childs}
        \label{fig:fig_appB}

\end{figure}

\begin{figure}[htbp]
    \centering
    
        \includegraphics[width=\textwidth]{./FIGURES/TWFE/covid_twfe_C_bysibs_elm_all_gpa_m_adj_Tsiblings_Soldest_4.pdf}
        \caption{Change in gap between children with siblings and only childs}
        \label{fig:fig_appC}

\end{figure}


\begin{figure}[htbp]
    \centering
    
        \includegraphics[width=\textwidth]{./FIGURES/TWFE/covid_twfe_D_bysibs_elm_all_gpa_m_adj_Tsiblings_Soldest_4.pdf}
        \caption{Change in gap between children with siblings and only childs}
        \label{fig:fig_appD}

\end{figure}


\makebox[0.1\width][l]{
\resizebox{\textwidth}{!}{
\begin{tabular}{lccc}
\toprule
\cmidrule(lr){2-4}
& \multicolumn{3}{c}{Standardized GPA}
\cmidrule(lr){2-4}
& Pre-Covid & Covid & Post-Covid  \\
& 2018-2019 & 2020-2021 & 2022-2023  \\
\cmidrule(lr){2-2} \cmidrule(lr){3-3} \cmidrule(lr){4-4}
& (1) & (2) & (3)  \\
\bottomrule
&  &  &   \\
Delay School (After SSA)&      -0.001   &      -0.010   &      -0.004   \\
                    &     (0.006)   &     (0.006)   &     (0.006)   \\
Local Linear        &         Yes   &         Yes   &         Yes   \\
                    &               &               &               \\
Observations        &     418,987   &     371,966   &     485,601   \\
Counterfactual mean &       0.043   &       0.020   &       0.031   \\
Bandwidth           &         365   &         365   &         365   \\
 

\bottomrule
\end{tabular}
}
}


\newpage

\makeatletter
\@ifclassloaded{beamer}{%
       \centering
       \resizebox{0.6\textwidth}{!}%
}{%
       \begin{table}[!tbp]\centering\def\sym#1{\ifmmode^{#1}\else\(^{#1}\)\fi}
       \centering
       \caption{Effects of younger sibling delaying school on older sibling standardized exams - 2 - m - a -  - 365}
       \label{tab:rd_summ_2_m_a_365}
       \resizebox{0.95\textwidth}{!}%
}
{
\makeatother
\makebox[0.1\width][l]{
\resizebox{\textwidth}{!}{
\begin{tabular}{lccc}
\toprule
\cmidrule(lr){2-4}
& \multicolumn{3}{c}{Standardized GPA} \\
\cmidrule(lr){2-4}
& Pre-Covid & Covid & Post-Covid  \\
& 2018-2019 & 2020-2021 & 2022-2023  \\
\cmidrule(lr){2-2} \cmidrule(lr){3-3} \cmidrule(lr){4-4}
& (1) & (2) & (3)  \\
\bottomrule
&  &  &   \\
\multirow{2}{*}{\shortstack[l]{Younger sibling born after \\ school-entry cutoff}}&      -0.000   &      -0.008   &       0.001   \\
                    &     (0.008)   &     (0.007)   &     (0.007)   \\
Local Linear        &         Yes   &         Yes   &         Yes   \\
                    &               &               &               \\
Observations        &     268,561   &     288,245   &     357,788   \\
Counterfactual mean &       0.053   &       0.027   &       0.053   \\
Bandwidth           &         365   &         365   &         365   \\
 

\bottomrule
\end{tabular}
}
}


\makeatletter
\@ifclassloaded{beamer}{%
       \centering
       \resizebox{0.6\textwidth}{!}%
}{%
       \begin{table}[!tbp]\centering\def\sym#1{\ifmmode^{#1}\else\(^{#1}\)\fi}
       \centering
       \caption{TWFE on 6th grade GPA by 2nd grade baseline resources}
       \label{tab:twfe_gpa_baseline_survey_1_pair1}
       \resizebox{0.65\textwidth}{!}%
}
{
\makeatother
\begin{tabular}{lccc}
\toprule
\cmidrule(lr){2-4}
& \multicolumn{3}{c}{TWFE} \\
\cmidrule(lr){2-4}
& 1 sibling & 2 siblings & 3 siblings  \\
\cmidrule(lr){2-2} \cmidrule(lr){3-3} \cmidrule(lr){4-4}
& (1) & (2) & (3)\\
\bottomrule
&  &  &  \\
&  &  &   \\
\multicolumn{4}{l}{\textit{Panel A: All studentes}} \\
\hspace{3mm}Mathematics&      -0.046***&      -0.078***&      -0.077** \\
                    &     (0.013)   &     (0.018)   &     (0.035)   \\
 
%&  &  &   \\
\hspace{3mm}Reading &      -0.023*  &      -0.035*  &      -0.072** \\
                    &     (0.013)   &     (0.018)   &     (0.035)   \\
                    &               &               &               \\
\hspace{3mm}Observations&     108,585   &      85,464   &      72,938   \\
 
&  &  &   \\
\multicolumn{4}{l}{\textit{Panel B: Low SES Households (Q1)}} \\
\hspace{3mm}Mathematics&      -0.026   &      -0.029   &      -0.166***\\
                    &     (0.026)   &     (0.033)   &     (0.057)   \\
 
%&  &  &   \\
\hspace{3mm}Reading &       0.001   &      -0.001   &      -0.133** \\
                    &     (0.026)   &     (0.034)   &     (0.058)   \\
                    &               &               &               \\
\hspace{3mm}Observations&      25,600   &      20,717   &      17,069   \\
 
&  &  &   \\
\multicolumn{4}{l}{\textit{Panel C: High SES Households (Q4)}} \\
\hspace{3mm}Mathematics&      -0.069** &       0.013   &       0.034   \\
                    &     (0.034)   &     (0.056)   &     (0.155)   \\
 
%&  &  &   \\
\hspace{3mm}Reading &      -0.026   &      -0.045   &       0.279*  \\
                    &     (0.034)   &     (0.056)   &     (0.153)   \\
                    &               &               &               \\
\hspace{3mm}Observations&      18,418   &      13,891   &      12,219   \\
 
&  &  &   \\
\multicolumn{4}{l}{\textit{Panel D: Households with no PC or Internet}} \\
\hspace{3mm}Mathematics&      -0.057***&      -0.113***&      -0.058   \\
                    &     (0.020)   &     (0.027)   &     (0.051)   \\
 
%&  &  &   \\
\hspace{3mm}Reading &      -0.036*  &      -0.033   &      -0.069   \\
                    &     (0.020)   &     (0.027)   &     (0.051)   \\
                    &               &               &               \\
\hspace{3mm}Observations&      46,281   &      36,305   &      30,041   \\
 
&  &  &   \\
\multicolumn{4}{l}{\textit{Panel E: Households with both PC and Internet}} \\
\hspace{3mm}Mathematics&      -0.013   &       0.015   &       0.307** \\
                    &     (0.034)   &     (0.056)   &     (0.146)   \\
 
%&  &  &   \\
\hspace{3mm}Reading &       0.009   &       0.006   &       0.505***\\
                    &     (0.035)   &     (0.056)   &     (0.147)   \\
                    &               &               &               \\
\hspace{3mm}Observations&      18,086   &      13,736   &      12,097   \\
 

\bottomrule
\end{tabular}
}
\@ifclassloaded{beamer}{%
}{%
       \end{table}
}

\input{./TABLES/twfe_gpa_baseline_survey_1_pair2.tex}
\makeatletter
\@ifclassloaded{beamer}{%
       \centering
       \resizebox{0.6\textwidth}{!}%
}{%
       \begin{table}[!tbp]\centering\def\sym#1{\ifmmode^{#1}\else\(^{#1}\)\fi}
       \centering
       \caption{TWFE on 7th grade GPA by 4th grade baseline resources}
       \label{tab:twfe_gpa_baseline_survey_1_pair3}
       \resizebox{0.65\textwidth}{!}%
}
{
\makeatother
\begin{tabular}{lccc}
\toprule
\cmidrule(lr){2-4}
& \multicolumn{3}{c}{TWFE} \\
\cmidrule(lr){2-4}
& 1 sibling & 2 siblings & 3 siblings  \\
\cmidrule(lr){2-2} \cmidrule(lr){3-3} \cmidrule(lr){4-4}
& (1) & (2) & (3)\\
\bottomrule
&  &  &  \\
&  &  &   \\
\multicolumn{4}{l}{\textit{Panel A: All studentes}} \\
\hspace{3mm}Mathematics&      -0.024***&      -0.070***&      -0.060***\\
                    &     (0.007)   &     (0.010)   &     (0.018)   \\
 
%&  &  &   \\
\hspace{3mm}Reading &      -0.021***&      -0.045***&      -0.033*  \\
                    &     (0.007)   &     (0.010)   &     (0.018)   \\
                    &               &               &               \\
\hspace{3mm}Observations&     365,702   &     292,698   &     254,104   \\
 
&  &  &   \\
\multicolumn{4}{l}{\textit{Panel B: Low SES Households (Q1)}} \\
\hspace{3mm}Mathematics&      -0.000   &      -0.020   &      -0.005   \\
                    &     (0.014)   &     (0.017)   &     (0.028)   \\
 
%&  &  &   \\
\hspace{3mm}Reading &      -0.012   &      -0.006   &       0.004   \\
                    &     (0.014)   &     (0.017)   &     (0.028)   \\
                    &               &               &               \\
\hspace{3mm}Observations&      90,252   &      75,485   &      63,910   \\
 
&  &  &   \\
\multicolumn{4}{l}{\textit{Panel C: High SES Households (Q4)}} \\
\hspace{3mm}Mathematics&      -0.026*  &      -0.091***&      -0.106   \\
                    &     (0.016)   &     (0.027)   &     (0.071)   \\
 
%&  &  &   \\
\hspace{3mm}Reading &      -0.031** &      -0.097***&      -0.066   \\
                    &     (0.016)   &     (0.027)   &     (0.072)   \\
                    &               &               &               \\
\hspace{3mm}Observations&      73,259   &      56,234   &      50,652   \\
 
&  &  &   \\
\multicolumn{4}{l}{\textit{Panel D: Households with no PC or Internet}} \\
\hspace{3mm}Mathematics&      -0.018   &      -0.107***&      -0.084*  \\
                    &     (0.013)   &     (0.020)   &     (0.046)   \\
 
%&  &  &   \\
\hspace{3mm}Reading &      -0.021   &      -0.077***&      -0.107** \\
                    &     (0.013)   &     (0.020)   &     (0.047)   \\
                    &               &               &               \\
\hspace{3mm}Observations&     113,464   &      91,731   &      79,607   \\
 
&  &  &   \\
\multicolumn{4}{l}{\textit{Panel E: Households with both PC and Internet}} \\
\hspace{3mm}Mathematics&      -0.007   &      -0.023   &      -0.003   \\
                    &     (0.012)   &     (0.018)   &     (0.040)   \\
 
%&  &  &   \\
\hspace{3mm}Reading &       0.006   &      -0.030*  &      -0.016   \\
                    &     (0.012)   &     (0.018)   &     (0.040)   \\
                    &               &               &               \\
\hspace{3mm}Observations&     136,957   &     108,035   &      92,888   \\
 

\bottomrule
\end{tabular}
}
\@ifclassloaded{beamer}{%
}{%
       \end{table}
}

\makeatletter
\@ifclassloaded{beamer}{%
       \centering
       \resizebox{0.6\textwidth}{!}%
}{%
       \begin{table}[!tbp]\centering\def\sym#1{\ifmmode^{#1}\else\(^{#1}\)\fi}
       \centering
       \caption{TWFE on GPA by baseline resources}
       \label{tab:twfe_ece_survey_1_pair4}
       \resizebox{0.65\textwidth}{!}%
}
{
\makeatother
\begin{tabular}{lccc}
\toprule
\cmidrule(lr){2-4}
& \multicolumn{3}{c}{TWFE} \\
\cmidrule(lr){2-4}
& 1 sibling & 2 siblings & 3 siblings  \\
\cmidrule(lr){2-2} \cmidrule(lr){3-3} \cmidrule(lr){4-4}
& (1) & (2) & (3)\\
\bottomrule
&  &  &  \\
&  &  &   \\
\multicolumn{4}{l}{\textit{Panel A: All studentes}} \\
\hspace{3mm}Mathematics&      -0.032***&      -0.053***&      -0.074***\\
                    &     (0.006)   &     (0.008)   &     (0.014)   \\
 
%&  &  &   \\
\hspace{3mm}Reading &      -0.018***&      -0.035***&      -0.045***\\
                    &     (0.006)   &     (0.008)   &     (0.014)   \\
                    &               &               &               \\
\hspace{3mm}Observations&     466,128   &     384,809   &     339,142   \\
 
&  &  &   \\
\multicolumn{4}{l}{\textit{Panel B: Low SES Households (Q1)}} \\
\hspace{3mm}Mathematics&      -0.018   &      -0.023   &      -0.044** \\
                    &     (0.012)   &     (0.015)   &     (0.021)   \\
 
%&  &  &   \\
\hspace{3mm}Reading &      -0.002   &      -0.006   &      -0.045** \\
                    &     (0.012)   &     (0.015)   &     (0.022)   \\
                    &               &               &               \\
\hspace{3mm}Observations&     119,170   &     103,085   &      90,231   \\
 
&  &  &   \\
\multicolumn{4}{l}{\textit{Panel C: High SES Households (Q4)}} \\
\hspace{3mm}Mathematics&      -0.039***&      -0.074***&      -0.091*  \\
                    &     (0.013)   &     (0.021)   &     (0.050)   \\
 
%&  &  &   \\
\hspace{3mm}Reading &      -0.026*  &      -0.057***&      -0.027   \\
                    &     (0.013)   &     (0.022)   &     (0.050)   \\
                    &               &               &               \\
\hspace{3mm}Observations&      91,916   &      72,508   &      64,890   \\
 
&  &  &   \\
\multicolumn{4}{l}{\textit{Panel D: Households with no PC or Internet}} \\
\hspace{3mm}Mathematics&      -0.032***&      -0.059***&      -0.063** \\
                    &     (0.009)   &     (0.014)   &     (0.029)   \\
 
%&  &  &   \\
\hspace{3mm}Reading &      -0.021** &      -0.036** &       0.004   \\
                    &     (0.010)   &     (0.014)   &     (0.030)   \\
                    &               &               &               \\
\hspace{3mm}Observations&     186,154   &     149,785   &     133,378   \\
 
&  &  &   \\
\multicolumn{4}{l}{\textit{Panel E: Households with both PC and Internet}} \\
\hspace{3mm}Mathematics&      -0.031***&      -0.042***&      -0.095***\\
                    &     (0.010)   &     (0.013)   &     (0.020)   \\
 
%&  &  &   \\
\hspace{3mm}Reading &      -0.005   &      -0.030** &      -0.066***\\
                    &     (0.011)   &     (0.013)   &     (0.021)   \\
                    &               &               &               \\
\hspace{3mm}Observations&     153,436   &     130,307   &     113,257   \\
 

\bottomrule
\end{tabular}
}
\@ifclassloaded{beamer}{%
}{%
       \end{table}
}


\makeatletter
\@ifclassloaded{beamer}{%
       \centering
       \resizebox{0.6\textwidth}{!}%
}{%
       \begin{table}[!tbp]\centering\def\sym#1{\ifmmode^{#1}\else\(^{#1}\)\fi}
       \centering
       \caption{WFE on GPA by baseline achievement and expectations}
       \label{tab:twfe_gpa_baseline_survey_2_pair1}
       \resizebox{0.65\textwidth}{!}%
}
{
\makeatother
\begin{tabular}{lccc}
\toprule
\cmidrule(lr){2-4}
& \multicolumn{3}{c}{TWFE} \\
\cmidrule(lr){2-4}
& 1 sibling & 2 siblings & 3 siblings  \\
\cmidrule(lr){2-2} \cmidrule(lr){3-3} \cmidrule(lr){4-4}
& (1) & (2) & (3)\\
\bottomrule
&  &  &  \\
&  &  &   \\
\multicolumn{4}{l}{\textit{Panel A: All studentes}} \\
\hspace{3mm}Mathematics&      -0.046***&      -0.078***&      -0.077** \\
                    &     (0.013)   &     (0.018)   &     (0.035)   \\
 
%&  &  &   \\
\hspace{3mm}Reading &      -0.023*  &      -0.035*  &      -0.072** \\
                    &     (0.013)   &     (0.018)   &     (0.035)   \\
                    &               &               &               \\
\hspace{3mm}Observations&     108,585   &      85,464   &      72,938   \\
 
&  &  &   \\
\multicolumn{4}{l}{\textit{Panel B: Student in bottom quartile of achievement}} \\
\hspace{3mm}Mathematics&      -0.028   &      -0.073*  &      -0.160** \\
                    &     (0.032)   &     (0.041)   &     (0.072)   \\
 
%&  &  &   \\
\hspace{3mm}Reading &       0.018   &      -0.026   &      -0.095   \\
                    &     (0.032)   &     (0.042)   &     (0.073)   \\
                    &               &               &               \\
\hspace{3mm}Observations&      14,632   &      11,987   &      10,145   \\
 
&  &  &   \\
\multicolumn{4}{l}{\textit{Panel C: Student in top quartile of achievement}} \\
\hspace{3mm}Mathematics&      -0.061** &      -0.103***&       0.030   \\
                    &     (0.026)   &     (0.038)   &     (0.083)   \\
 
%&  &  &   \\
\hspace{3mm}Reading &      -0.049*  &      -0.044   &       0.009   \\
                    &     (0.026)   &     (0.038)   &     (0.084)   \\
                    &               &               &               \\
\hspace{3mm}Observations&      34,500   &      26,134   &      22,113   \\
 
&  &  &   \\
\multicolumn{4}{l}{\textit{Panel D: Max Expectation: Finish school}} \\
\hspace{3mm}Mathematics&       0.004   &      -0.078   &      -0.222   \\
                    &     (0.063)   &     (0.083)   &     (0.141)   \\
 
%&  &  &   \\
\hspace{3mm}Reading &       0.037   &       0.004   &      -0.250*  \\
                    &     (0.064)   &     (0.085)   &     (0.145)   \\
                    &               &               &               \\
\hspace{3mm}Observations&       5,127   &       4,075   &       3,422   \\
 
&  &  &   \\
\multicolumn{4}{l}{\textit{Panel E: Max Expectation: 4-year college or grad school}} \\
\hspace{3mm}Mathematics&      -0.048***&      -0.073***&      -0.041   \\
                    &     (0.015)   &     (0.021)   &     (0.042)   \\
 
%&  &  &   \\
\hspace{3mm}Reading &      -0.030** &      -0.031   &      -0.041   \\
                    &     (0.015)   &     (0.021)   &     (0.043)   \\
                    &               &               &               \\
\hspace{3mm}Observations&      87,535   &      67,871   &      57,831   \\
 

\bottomrule
\end{tabular}
}
\@ifclassloaded{beamer}{%
}{%
       \end{table}
}

\makeatletter
\@ifclassloaded{beamer}{%
       \centering
       \resizebox{0.6\textwidth}{!}%
}{%
       \begin{table}[!tbp]\centering\def\sym#1{\ifmmode^{#1}\else\(^{#1}\)\fi}
       \centering
       \caption{}
       \label{tab:twfe_gpa_baseline_survey_2_pair2}
       \resizebox{0.65\textwidth}{!}%
}
{
\makeatother
\begin{tabular}{lccc}
\toprule
\cmidrule(lr){2-4}
& \multicolumn{3}{c}{TWFE} \\
\cmidrule(lr){2-4}
& 1 sibling & 2 siblings & 3 siblings  \\
\cmidrule(lr){2-2} \cmidrule(lr){3-3} \cmidrule(lr){4-4}
& (1) & (2) & (3)\\
\bottomrule
&  &  &  \\
&  &  &   \\
\multicolumn{4}{l}{\textit{Panel A: All studentes}} \\
\hspace{3mm}Mathematics&      -0.024***&      -0.062***&      -0.110***\\
                    &     (0.007)   &     (0.010)   &     (0.019)   \\
 
%&  &  &   \\
\hspace{3mm}Reading &      -0.014** &      -0.064***&      -0.077***\\
                    &     (0.007)   &     (0.010)   &     (0.020)   \\
                    &               &               &               \\
\hspace{3mm}Observations&     341,265   &     272,263   &     236,637   \\
 
&  &  &   \\
\multicolumn{4}{l}{\textit{Panel B: Student in bottom quartile of achievement}} \\
\hspace{3mm}Mathematics&      -0.013   &      -0.052***&      -0.084** \\
                    &     (0.015)   &     (0.020)   &     (0.035)   \\
 
%&  &  &   \\
\hspace{3mm}Reading &      -0.005   &      -0.049** &      -0.078** \\
                    &     (0.015)   &     (0.020)   &     (0.035)   \\
                    &               &               &               \\
\hspace{3mm}Observations&      64,124   &      52,974   &      45,942   \\
 
&  &  &   \\
\multicolumn{4}{l}{\textit{Panel C: Student in top quartile of achievement}} \\
\hspace{3mm}Mathematics&      -0.038** &      -0.094***&      -0.098** \\
                    &     (0.015)   &     (0.022)   &     (0.048)   \\
 
%&  &  &   \\
\hspace{3mm}Reading &      -0.022   &      -0.071***&      -0.095*  \\
                    &     (0.015)   &     (0.023)   &     (0.049)   \\
                    &               &               &               \\
\hspace{3mm}Observations&      97,944   &      74,571   &      64,319   \\
 
&  &  &   \\
\multicolumn{4}{l}{\textit{Panel D: Max Expectation: Finish school}} \\
\hspace{3mm}Mathematics&      -0.021   &      -0.064*  &       0.062   \\
                    &     (0.030)   &     (0.039)   &     (0.065)   \\
 
%&  &  &   \\
\hspace{3mm}Reading &      -0.004   &      -0.082** &      -0.018   \\
                    &     (0.030)   &     (0.039)   &     (0.065)   \\
                    &               &               &               \\
\hspace{3mm}Observations&      22,087   &      18,509   &      15,822   \\
 
&  &  &   \\
\multicolumn{4}{l}{\textit{Panel E: Max Expectation: 4-year college or grad school}} \\
\hspace{3mm}Mathematics&      -0.028***&      -0.061***&      -0.136***\\
                    &     (0.008)   &     (0.012)   &     (0.024)   \\
 
%&  &  &   \\
\hspace{3mm}Reading &      -0.018** &      -0.062***&      -0.106***\\
                    &     (0.008)   &     (0.012)   &     (0.024)   \\
                    &               &               &               \\
\hspace{3mm}Observations&     270,591   &     212,753   &     184,893   \\
 

\bottomrule
\end{tabular}
}
\@ifclassloaded{beamer}{%
}{%
       \end{table}
}

\makeatletter
\@ifclassloaded{beamer}{%
       \centering
       \resizebox{0.6\textwidth}{!}%
}{%
       \begin{table}[!tbp]\centering\def\sym#1{\ifmmode^{#1}\else\(^{#1}\)\fi}
       \centering
       \caption{TWFE on 7th grade GPA by 4th grade baseline achievement and expectations}
       \label{tab:twfe_gpa_baseline_survey_2_pair3}
       \resizebox{0.65\textwidth}{!}%
}
{
\makeatother
\begin{tabular}{lccc}
\toprule
\cmidrule(lr){2-4}
& \multicolumn{3}{c}{TWFE} \\
\cmidrule(lr){2-4}
& 1 sibling & 2 siblings & 3 siblings  \\
\cmidrule(lr){2-2} \cmidrule(lr){3-3} \cmidrule(lr){4-4}
& (1) & (2) & (3)\\
\bottomrule
&  &  &  \\
&  &  &   \\
\multicolumn{4}{l}{\textit{Panel A: All studentes}} \\
\hspace{3mm}Mathematics&      -0.024***&      -0.070***&      -0.060***\\
                    &     (0.007)   &     (0.010)   &     (0.018)   \\
 
%&  &  &   \\
\hspace{3mm}Reading &      -0.021***&      -0.045***&      -0.033*  \\
                    &     (0.007)   &     (0.010)   &     (0.018)   \\
                    &               &               &               \\
\hspace{3mm}Observations&     365,702   &     292,698   &     254,104   \\
 
&  &  &   \\
\multicolumn{4}{l}{\textit{Panel B: Student in bottom quartile of achievement}} \\
\hspace{3mm}Mathematics&       0.022   &      -0.009   &      -0.101***\\
                    &     (0.015)   &     (0.019)   &     (0.033)   \\
 
%&  &  &   \\
\hspace{3mm}Reading &       0.020   &      -0.005   &      -0.015   \\
                    &     (0.015)   &     (0.020)   &     (0.034)   \\
                    &               &               &               \\
\hspace{3mm}Observations&      76,396   &      63,590   &      55,311   \\
 
&  &  &   \\
\multicolumn{4}{l}{\textit{Panel C: Student in top quartile of achievement}} \\
\hspace{3mm}Mathematics&      -0.046***&      -0.118***&      -0.166***\\
                    &     (0.014)   &     (0.021)   &     (0.045)   \\
 
%&  &  &   \\
\hspace{3mm}Reading &      -0.042***&      -0.081***&      -0.062   \\
                    &     (0.014)   &     (0.021)   &     (0.045)   \\
                    &               &               &               \\
\hspace{3mm}Observations&     100,921   &      76,928   &      66,386   \\
 
&  &  &   \\
\multicolumn{4}{l}{\textit{Panel D: Max Expectation: Finish school}} \\
\hspace{3mm}Mathematics&       0.018   &      -0.033   &      -0.035   \\
                    &     (0.029)   &     (0.037)   &     (0.060)   \\
 
%&  &  &   \\
\hspace{3mm}Reading &      -0.085***&      -0.057   &      -0.051   \\
                    &     (0.029)   &     (0.037)   &     (0.060)   \\
                    &               &               &               \\
\hspace{3mm}Observations&      26,308   &      22,144   &      19,072   \\
 
&  &  &   \\
\multicolumn{4}{l}{\textit{Panel E: Max Expectation: 4-year college or grad school}} \\
\hspace{3mm}Mathematics&      -0.032***&      -0.087***&      -0.092***\\
                    &     (0.008)   &     (0.011)   &     (0.022)   \\
 
%&  &  &   \\
\hspace{3mm}Reading &      -0.024***&      -0.057***&      -0.076***\\
                    &     (0.008)   &     (0.011)   &     (0.023)   \\
                    &               &               &               \\
\hspace{3mm}Observations&     287,508   &     226,685   &     196,682   \\
 

\bottomrule
\end{tabular}
}
\@ifclassloaded{beamer}{%
}{%
       \end{table}
}

\input{./TABLES/twfe_gpa_baseline_survey_2_pair4.tex}


\begin{comment}

\clearpage

\setcounter{figure}{0}
\renewcommand\thefigure{B.\arabic{figure}}    

\setcounter{table}{0}
\renewcommand{\thetable}{B.\arabic{table}}
\setcounter{subsection}{0}

\section*{Appendix B: Robustness} \label{sec:appB}

%\section{Robustness}\label{sec:robustness}

\subsection{Potential sample selection}

Discuss how siblings are observed. We only see observations as long as they are enrolled in data. Most children are enrolled in primary school \textcolor{green}{(show data from household surveys)}. However, most children are not enrolled until 3-4 years old. Since our last year of administrative data is 2024, this means that some families with 2 children, one of which was born after 2021, might be observed as only childs. To address any concerns from this selection I do the following:

\begin{itemize}
    \item These families are not particularly different from others? (this true)?
    \item I use data up to 2023 and 2022 to define families (and potentially imposing a similar bias to earlier years) and still see results change around COVID. \textcolor{green}{Pending Analyis}
\end{itemize}


\subsection{Siblings in same vs different schools}

\subsection{Different definitions of siblings: father, both, caretaker}

\subsection{Only children: Missreport}

What if they are mostly students leaving by themselves? And not reporting parent's IDs? How can we validate this?

\subsection{Younger and middle child}

\subsection{Unadjusted scores}


\clearpage


\setcounter{figure}{0}
\renewcommand\thefigure{C.\arabic{figure}}    

\setcounter{table}{0}
\renewcommand{\thetable}{C.\arabic{table}}
\setcounter{subsection}{0}

\section*{Appendix C: Heterogeneity} \label{sec:appC}

Surprisingly, results are consistent across...

\subsection{Reinforcement vs Compensation}



\clearpage

\setcounter{figure}{0}
\renewcommand\thefigure{D.\arabic{figure}}    

\setcounter{table}{0}
\renewcommand{\thetable}{D.\arabic{table}}
\setcounter{subsection}{0}

\section*{Appendix D: Validating Sibling Identification} \label{sec:appD}

\subsection{Contrasting with survey responses}


\subsection{Number of siblings}

In \textcolor{green}{XX} one survey question asked about number of siblings.


\subsection{Number of people in the household}

In the 2nd grade survey in 2015 and 2016 a question asked about the number of adults and children in the household. In \textcolor{green}{XX} I show the distribution of responses by number of children estimated with matching parent IDs. 

\subsection{Potential composition of sample}

By using older siblings, it is less and less likely to see them in lower grades for later years. For example, to see an older sibling with 3 younger siblings in 2024 in first grade, they would have to have siblings within 3 years below them, otherwise they wouldn't be seen in the data. These creates two isses. First, the TWFE might be overweighting school closure years over years after schools re-opened. We address this with the time fixed effect. Another issue is that these families are likely different given the short age gap between all the children. However, this issue is less prevalent in later grades. The fact that I see similar pattern in 1st as in 6th grade assuages concerns for this.



\newpage

\end{comment}